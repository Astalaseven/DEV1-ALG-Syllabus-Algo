% ====================
\chapter{Les chaines}
% ====================

	Dans ce chapitre, 
	nous allons apprendre à manipuler du texte 
	en étudiant les différentes fonctions associées 
	aux variables de type chaine et caractère. 
	Ce sera aussi l'occasion de revoir quelques techniques algorithmiques 
	déjà acquises pour les nombres (alternatives, boucles) 
	afin de les consolider dans un contexte plus \og littéraire \fg.
	
\section{Introduction}
% ====================

	Nous avons défini au chapitre 2 deux types de variables 
	qui permettent de stocker du texte : 
	le caractère 
	(pour les différentes lettres et symboles 
	qui apparaissent sur le clavier de votre ordinateur, 
	par exemple 'a', 'B', '?', '3', '@', etc.) 
	et la chaine (qui est un assemblage de plusieurs caractères) 
	comme par exemple "Nous voici déjà arrivés au chapitre 7 !".

	Observez la petite nuance d'écriture :
	un caractère sera entouré de simples guillemets (' ') 
	et une chaine de doubles guillemets (" ").

	La taille (ou longueur) d'une chaine 
	est le nombre de caractères qu'elle contient. 
	Remarquez qu'une chaine peut être vide, mais pas un caractère !
	Ne confondez pas la chaine vide ("", de taille nulle) 
	avec le caractère blanc (' ') qui contient l'espace séparant 
	2 mots d'un texte et que vous obtenez en enfonçant 
	la touche d'espacement au bas du clavier.

	La chaine (string en anglais) 
	est un type à part entière dans la plupart des langages informatiques, 
	bien que certains langages utilisent d'autres artifices 
	pour simuler les chaines 
	(par exemple le C où la chaine s'apparente plus 
	à un tableau de caractères). 
	Au début de l'histoire de l'informatique, 
	les chaines étaient limitées à 255 caractères ;
	actuellement, la majorité des langages admettent des chaines 
	de longueur 65535 ou plus 
	(soit suffisament pour contenir tout un chapitre de ce syllabus). 
	La plupart des langages orienté objet offrent une classe 
	dédiée à la représentation et la manipulation des chaines de caractères.

	La manipulation de chaines peut s'avérer être 
	une source de complications pour le programmeur, 
	en particulier pour le développeur de sites internet 
	en raison des problèmes de compatibilité des différents systèmes informatiques, 
	des différentes normes de représentations 
	et des différents alphabets utilisés à travers le monde. 
	Dans tout ce qui suit, 
	nous simplifierons sensiblement cette problématique 
	(même au niveau du français) 
	en ne considérant pas les problèmes liés aux différentes 
	accentuations des lettres (accents, trémas, cédilles\dots). 

\section{Manipuler les caractères}
% ================================

	Introduisons quelques fonctions (ou primitives) 
	agissant sur les caractères. 
	Leur écriture s'apparente à celle de modules 
	que vous pourrez utiliser tels quels dans les exercices. 
	Nous ne détaillons pas leur code ici, 
	car il fait intervenir des aspects techniques 
	extérieurs au cadre du cours d'algorithmique 
	(codage des caractères, code ASCII, normes ISO\dots).

	\begin{center}
		\pseudocode{estLettre(car: caractère) \Gives~booléen}
	\end{center}

	Cette fonction indique si un caractère est une lettre. 
	Par exemple elle retourne vrai pour 'a', 'e', 'G', 'K', 
	mais faux pour '4', '\$', '@'\dots %$ 
		
	Si on veut savoir si une lettre est une minuscule ou majuscule, 
	on utilisera les fonctions analogues

	\begin{center}
		\pseudocode{estMinuscule(car: caractère) \Gives~booléen}
	\end{center}
	
	et

	\begin{center}
		\pseudocode{estMajuscule(car: caractère) \Gives~booléen}
	\end{center}

	Il va de soi que si \textit{car} n'est pas une lettre, 
	ces deux fonctions retournent faux.

	\begin{center}
		\pseudocode{estChiffre(car: caractère) \Gives~booléen}
	\end{center}

	Cette fonction permet de savoir si un caractère est un chiffre. 
	Elle retourne vrai uniquement pour les dix caractères 
	'0', '1', '2', '3', '4', '5', '6', '7', '8' et '9' 
	et faux dans tous les autres cas.

	On peut aussi convertir une majuscule en minuscule, grâce à la fonction :

	\begin{center}
		\pseudocode{majuscule(car: caractère) \Gives~caractère}
	\end{center}

	Par exemple, si \textit{car} vaut 'h', 
	cette fonction retourne 'H'. 
	L'opération inverse se fait avec :

	\begin{center}
		\pseudocode{minuscule(car: caractère) \Gives~caractère}
	\end{center}

	Il peut aussi être pratique de connaitre 
	la position d'une lettre dans l'alphabet. 
	Ceci se fera à l'aide de la fonction :

	\begin{center}
		\pseudocode{numLettre(car: caractère) \Gives~entier}
	\end{center}

	qui retourne toujours un entier entre 1 et 26. 
	Par exemple \pseudocode{numLettre('E')} donnera 5, 
	ainsi que \pseudocode{numLettre('e')}, 
	cette fonction traite donc de la même manière 
	les majuscules et les minuscules. 
	En vertu de ce qui a été écrit plus haut, 
	numLettre retournera aussi 5 pour les caractères 'é', 'è', 'ê', 'ë'\dots). 
	N.B. :  attention, il est interdit d'utiliser cette fonction 
	si le caractère n'est pas une lettre !

	Il peut être utile d'avoir un outil qui fait l'opération inverse, 
	à savoir associer la lettre de l'alphabet correspondant à une position donnée. 
	Pour cela, nous aurons : 

	\begin{center}
		\pseudocode{lettreMaj(n: entier) \Gives~caractère}
	\end{center}
	
	et

	\begin{center}
		\pseudocode{lettreMin(n: entier) \Gives~caractère}
	\end{center}

	qui retournent respectivement 
	la forme majuscule ou minuscule de la n-ème lettre de l'alphabet 
	(où \textit{n} sera obligatoirement compris entre 1 et 26). 
	Par exemple, \pseudocode{lettreMaj(13)} retourne 'M' 
	tandis que \pseudocode{lettreMin(26)} retourne 'z'.

\section{Convertir en chaine}
% ===========================

	Un caractère peut être vu comme une chaine de taille 1, 
	mais techniquement, 
	il est important de faire la distinction entre ces deux types. 
	Si on veut transformer un caractère en chaine, 
	on utilisera la fonction :

	\begin{center}
		\pseudocode{chaine(car: caractère) \Gives~chaine}
	\end{center}

	Par exemple, si \textit{car} vaut 'a', cette fonction retournera "a". 
	Il est cependant admissible d'utiliser le simple signe d'affectation 
	qui réalise la conversion de façon automatique :

	\begin{center}
		\pseudocode{varChaine \Gets~varCaractère}
	\end{center}

	De façon similaire, 
	on pourra transformer une variable numérique en chaine avec :

	\begin{center}
		\pseudocode{chaine(n : entier) \Gives~chaine}
	\end{center}

	ou

	\begin{center}
		\pseudocode{chaine(x : réel) \Gives~chaine}
	\end{center}

	Ainsi, \pseudocode{chaine(23)} donnera "23" 
	et \pseudocode{chaine (3,14)} donnera "3,14". 
	Notez qu'on utilise le même nom de fonction dans ces derniers cas 
	(concept de « surcharge » qui sera développé 
	dans le chapitre orienté objet du cours \og Algorithmique II \fg). 
	La différence entre ces différentes fonctions 
	se fait par le type de la variable en entrée.

	On peut aussi transformer une chaine 
	contenant des caractères numériques 
	en nombre par la fonction inverse :

	\begin{center}
		\pseudocode{nombre(ch : chaine) \Gives~réel}
	\end{center}

	Ainsi, 
	\pseudocode{nombre("3,14")} retournera 3,14. 
	Notez que si \textit{n} est un entier, 
	l'instruction \pseudocode{n \Gets~nombre("3,14")} affectera \textit{n} à 3, 
	conformément à la remarque du paragraphe \vref{affinterne}. 
	On supposera encore que si la chaine \textit{ch} ne représente pas 
	un nombre de façon correcte, la fonction retournera 0.

\section{Manipuler les chaines}
% =============================

	Venons-en maintenant aux différentes fonctions 
	qui permettent de manipuler les chaines ; 
	ces fonctions ont des formes équivalentes 
	dans la plupart des langages de programmation. 
	Nous adapterons pour le pseudo-code 
	les fonctions les plus courantes et les plus utiles.
	
	\subsection{Longueur d'une chaine}
	%---------------------------------

		La longueur d'une chaine (ou encore sa taille) 
		est le nombre de caractères qu'elle contient. 
		Nous pourrons l'évaluer avec la fonction

		\begin{center}
			\pseudocode{longueur(ch : chaine) \Gives~entier}
		\end{center}
	
		Par exemple : \pseudocode{longueur("algorithmique")} retourne 13 
		et \pseudocode{longueur("")} retourne 0.

	\subsection{Accès aux différents caractères d'une chaine}
	%--------------------------------------------------------

		La fonction

		\begin{center}
			\pseudocode{car(ch: chaine, n: entier) \Gives~caractère}
		\end{center}

		permet de connaitre le n-ème caractère contenu dans une chaine. 
		Pour une utilisation correcte de cette fonction, 
		il faut que la valeur de \textit{n} soit au moins égale à 1 
		et qu'elle ne dépasse pas la longueur de la chaine. 
		Par exemple \pseudocode{car("Spirou", 4)} retourne 'r', 
		mais \pseudocode{car("Spirou", 8)} générera un message d'erreur.

	\subsection{Extraction de sous-chaine}
	%-------------------------------------

		Cette fonction importante 
		permet d'extraire une portion 
		d'une certaine longueur d'une chaine donnée, 
		et ceci à partir d'une position donnée. 
		L'en-tête de cette fonction 
		(qui reçoit donc 3 paramètres entrants) est la suivante :

		\begin{center}
			\pseudocode{sousChaine(ch: chaine, pos: entier, long: entier) \Gives~chaine}
		\end{center}

		Il faudra aussi être très vigilant 
		pour une utilisation correcte: 
		\pseudocode{pos} doit être compris entre 1 et la taille de la chaine, 
		et la valeur de \pseudocode{long} doit être telle 
		qu'on ne déborde pas de la chaine ! 

		Par exemple, \pseudocode{sousChaine("algorithmique", 5, 3)} 
		donne "rit".

		Cette fonction est très utile pour sélectionner 
		des portions d'une chaine 
		contenant des informations codées sous un certain format. 
		Prenons par exemple une date stockée dans une chaine \textit{ch} de format 
		"JJ/MM/AAAA" (de longueur 10). 
		Pour extraire le jour de cette date, 
		on écrira \pseudocode{sousChaine(ch, 1, 2)} ; 
		le mois s'obtiendra avec \pseudocode{sousChaine(ch, 4, 2)} 
		et l'année avec \pseudocode{sousChaine(ch, 7, 4)}. 
		Attention, les 3 résultats obtenus sont des chaines de chiffres 
		et non des nombres !

	\subsection{Recherche de sous-chaine}
	%------------------------------------
	
		Cette fonction permet de savoir 
		si une sous-chaine donnée (ou un caractère) 
		est présent dans une chaine donnée. 
		Elle permet d'éviter d'écrire 
		le code correspondant à une recherche : 

		\begin{center}
			\pseudocode{estDansChaine(ch: chaine, sous-chaine: chaine [ou caractère]) \Gives~entier}			
		\end{center}

		La valeur de l'entier renvoyé est la position 
		où commence la sous-chaine recherchée. 
		Par exemple, 
		\pseudocode{estDansChaine("algorithmique", "mi")} retournera 9. 
		Si la sous-chaine ne s'y trouve pas, 
		la fonction retourne 0. 
		On peut admettre ici d'écrire un caractère à la place de la sous-chaine. 
		Par exemple, 
		\pseudocode{estDansChaine("algorithmique", 'm')} 
		retournera également 9. 
		
		\subsection{Concaténation de chaines}
		%------------------------------------

			Il est fréquent de devoir rassembler plusieurs chaines 
			pour former une seule chaine plus grande, 
			il s'agit de l'opération de concaténation. 
			La syntaxe est la suivante :
	
			\begin{center}
				\pseudocode{concat(ch1, ch2, \dots, chN: chaine) \Gives~chaine}
			\end{center}
			
			On remarquera ici le cas exceptionnel 
			d'une fonction admettant un nombre indéfini de paramètres. 
			Par exemple, \pseudocode{concat("al", "go", "rithmique")}
			donnera "algorithmique". 
			On admettra aussi comme notation alternative 
			l'utilisation du signe « + » 
			qui est ici sans équivoque 
			entre des variables non numériques :

			\begin{center}
				\pseudocode{ch \Gets~ch1 + ch2 + \dots + chN}
			\end{center}

\section{Exercices}
%==================

\begin{Exercice}{Calcul de fraction}
	Écrire un algorithme 
	qui reçoit une fraction sous forme de chaine, 
	et retourne la valeur numérique de celle-ci. 
	Par exemple, si la fraction donnée est "5/8", 
	l'algorithme renverra 0,625. 
	On peut considérer que la fraction donnée est correcte, 
	elle est composée de 2 entiers séparés 
	par le caractère de division '/'.
\end{Exercice}

\begin{Exercice}{Conversion de nom}
	Écrire un algorithme 
	qui reçoit le nom complet d'une personne 
	dans une chaine sous la forme "nom, prénom" 
	et la renvoie au format "prénom nom" 
	(sans virgule séparatrice). 
	Exemple : "De Groote, Jan" deviendra "Jan De Groote".	
\end{Exercice}

\begin{Exercice}{Gauche et droite}
	Écrire un module \pseudocode{gauche} 
	et un \pseudocode{droite}
	qui retourne la chaine formée respectivement 
	des n premiers et des n derniers caractères d'une chaine donnée.	
\end{Exercice}

\begin{Exercice}{Grammaire}
	Écrire un algorithme 
	qui met un mot en « ou » au pluriel. 
	Pour rappel, 
	un mot en « ou » prend un « s » à l'exception des 7 mots 
	bijou, caillou, chou, genou, hibou, joujou et pou qui prennent 
	un « x » au pluriel. 
	Exemple : un clou, des clous, un hibou, des hiboux. 
	Si le mot soumis à l'algorithme n'est pas un mot en « ou », 
	un message adéquat sera affiché.
\end{Exercice}

\begin{Exercice}{Le code correct}
	Le code IBAN de votre compte en banque 
	est composé de 16 caractères alphanumériques 
	(2 lettres suivies de 14 chiffres). 
	Pour la lisibilité, 
	on l'écrit souvent en 4 paquets de 4 caractères 
	séparés par des blancs, par ex. : "BE89 6352 4390 0285" 
	(soit une chaine de 19 caractères au total). 
	On peut vérifier si un compte est valide par la manipulation suivante :
	le reste de la division par 97 du nombre formé 
	par les 10 chiffres entre le 5ème et 14ème caractères du code 
	doit donner le nombre formé par les 2 derniers chiffres. 
	On peut voir directement que, pour l'exemple donné, 
	896352439002 MOD 97 est bien égal à 85.

	Écrire l'algorithme qui reçoit un code IBAN sous la forme 
	d'une chaine de 19 caractères
	et retourne un booléen indiquant si ce code est correct.
\end{Exercice}

\begin{Exercice}{Email valide}
	Écrire un algorithme 
	qui vérifie si une chaine représentant une adresse email 
	est valide. 
	Pour simplifier, 
	nous dirons qu'une adresse email valide 
	est une chaine de la forme \verb|ch1@ch2.ch3|, 
	où les sous-chaines ch1 et ch2 ne peuvent contenir 
	que des caractères alpha-numériques, 
	et ch3 ne contient que 2 ou 3 lettres. 
	(Dans la réalité, c'est un peu plus complexe, 
	ch1 pouvant contenir d'autres caractères 
	tels que points, tirets, tirets bas (underscore)\dots).
\end{Exercice}

\begin{Exercice}{Les palindromes}
	Cet exercice a déjà été réalisé pour des entiers, 
	en voici 2 versions « chaine » ! 
	\begin{enumerate}[label=\alph*)]
	\item 
		Écrire un algorithme qui vérifie 
		si un mot donné sous forme de chaine 
		constitue un palindrome 
		(comme par exemple "kayak", "radar" ou "saippuakivikauppias" 
		(marchand de savon en finnois)
	\item
		Écrire un algorithme qui vérifie 
		si une phrase donnée sous forme de chaine constitue un palindrome 
		(comme par exemple "Esope reste ici et se repose" 
		ou "Tu l'as trop écrasé, César, ce Port-Salut !"). 
		Dans cette seconde version, 
		on fait abstraction des majuscules/minuscules 
		et on néglige les espaces et tout signe de ponctuation.
	\end{enumerate}
\end{Exercice}

\begin{Exercice}{Remplacement de sous-chaines}
	Écrire un algorithme 
	qui remplace dans une chaine donnée 
	toutes les sous-chaines ch1 par la sous-chaine ch2. 
	Attention, 
	cet exercice est plus coriace qu'il n'y parait à première vue\dots~
	Assurez-vous que votre code n'engendre pas de boucle infinie. 
\end{Exercice}

\begin{Exercice}{Conversion nombres binaires et décimaux}
	Réécrire les exercices de conversion de nombres binaires et décimaux
	du chapitre sur les boucles 
	en considérant que les nombres entrés et affichés sont des chaines. 
	D'après-vous, 
	en quoi cette transformation 
	est-elle une amélioration de la version purement numérique ?
\end{Exercice}

\begin{Exercice}{Normaliser une chaine}
	Écrire un module qui reçoit une chaine et retourne une autre chaine,
	version normalisée de la première.
	Par normalisée, on entend~:~enlever tout ce qui n'est pas une lettre 
	et tout mettre en majuscule.
	\\Exemple~:~"Le <COBOL>, c'est la santé~!" devient "LECOBOLCESTLASANTE".
\end{Exercice}

\begin{Exercice}{Le chiffre de César}
	\label{ex:cesar}
	Depuis l'antiquité, les hommes politiques, les militaires, 
	les hommes d'affaires cherchent à garder secret les messages
	importants qu'ils doivent envoyer.
	L'empereur César utilisait une technique (on dit un \emph{chiffrement})
	qui porte à présent son nom~:
	remplacer chaque lettre du message par la lettre qui se situe 
	$k$ position plus loin dans l'alphabet (cycliquement).
	
	Exemple~:~si $k$ vaut $2$, 
	alors le texte clair "CESAR" devient "EGUCT" lorsqu'il est chiffré 
	et le texte "ZUT" devient "BWV".
	
	Bien sûr, il faut que l'expéditeur du message et le récepteur
	se soient mis d'accord sur la valeur de $k$.
	
	On vous demande d'écrire un module qui reçoit une chaine ne contenant
	que des lettres majuscules ainsi que la valeur de $k$ et qui retourne
	la version chiffrée du message.

	On vous demande également d'écrire le module de déchiffrement.
	Ce module reçoit un message chiffré et la valeur de $k$ qui a été
	utilisée pour le chiffrer et retourne le message en clair.
	Attention~! Ce second module est \textbf{très simple}.

	%\begin{Solution}
		%\cadre{
		%\begin{Pseudocode}
			%\LComment{Chiffrer un message en utilisant le chiffre de César}
			%\Module{chiffrerCésar}{msgClair\In : chaine, déplacement\In : entier}{chaine}
				%\Decl msgChiffré : chaine
				%\Decl carClair, carChiffré : caractères
				%\Decl i : entier
				%\Empty
				%\Let msgChiffré \Gets ""
				%\For{i \K{de} 1 \K{à} long(msgClair)}
					%\Let carClair \Gets car(msgClair, i)
					%\Let carChiffré \Gets avancer(carClair, déplacement)
					%\Let msgChiffré \Gets concat( msgChiffré, chaine(carChiffré) )
				%\EndFor
				%\Return msgChiffré
			%\EndModule

			%\Empty
			%\LComment{Calcule la lettre qui est "delta" position plus loin dans l'alphabet (circulairement)}
			%\Module{avancer}{lettre\In : caractère, delta\In : entier}{caractère}
				%\Return lettre( (numLettre(lettre) + delta - 1) MOD 26 + 1 ) 
			%\EndModule

			%\Empty
			%\LComment{Déchiffrer un message chiffré avec le chiffre de César}
			%\Module{déchiffrerCésar}{msgClair\In : chaine, déplacement\In : entier}{chaine}
				%\Return chiffrerCésar(msgClair, 26 - déplacement)
			%\EndModule
		%\end{Pseudocode}
		%}
	%\end{Solution}
	
\end{Exercice}
