%==============================
\chapter{Spécifier le problème}
%==============================

	\marginicon{objectif}
	Comme nous l'avons dit, 
	un problème ne sera véritablement bien spécifié 
	que s’il s’inscrit dans le schéma suivant~:
		
	\medskip
	\begin{center}
	\begin{Ovalbox}
		{\textbf{étant donné} [les données] 
		\textbf{on demande} [résultat]}
	\end{Ovalbox}
	\end{center}
	\medskip
	
	La première étape dans la résolution d’un problème est de
	préciser ce problème à partir de l'énoncé,
	c-à-d de déterminer et préciser les données et le résultat.
	Il ne s’agit pas d’un travail facile 
	et c'est celui par lequel nous allons commencer.

	%----------------------------------------------
	\section{Déterminer les données et le résultat}
	%----------------------------------------------
	
		La toute première étape est de parvenir à extraire
		d'un énoncé de problème, quelles sont les données
		et quel est le résultat attendu%
		\footnote{%
			Plaçons-nous pour le moment dans le cadre
			de problèmes où il y a exactement un résultat.%
		}.
	
		\begin{Emphase}
			\paragraph{Exemple.}
			Soit l'énoncé suivant :
			\og
				Calculer la surface d'un rectangle 
				à partir de sa longueur et sa largeur
			\fg.
			
			Quelles sont les données ? Il y en a deux :	
			\begin{itemize}
				\item la longueur du rectangle ;
				\item sa largeur.
			\end{itemize}
		
			Quel est le résultat attendu ? la surface du rectangle.
		\end{Emphase}
		
	%----------------------------------------------
	\section{Les noms}\index{noms}
	%----------------------------------------------
	
		Pour identifier clairement chaque \textbf{donnée}
		et pouvoir y faire référence dans le futur algorithme
		nous devons lui attribuer un \textbf{nom}%		
		\footnote{%
				Dans ce cours, on va choisir des noms en Français,
				mais vous pouvez très bien choisir
				des noms Anglais si vous vous sentez
				suffisamment à l'aise avec cette langue.
		}.
		Il est important de bien choisir les noms. 
		Le but est de trouver un nom qui soit suffisamment court,
		tout en restant explicite et ne prêtant pas à confusion.
		
\clearpage
	
		\begin{Emphase}
			\paragraph{Exemple.}	
			Quel nom choisir pour la longueur d'un rectangle ?
	
			On peut envisager les noms suivants :
			\begin{itemize}
			\item
				\lda{longueur} est probablement le plus approprié.
			\item
				\lda{longueurRectangle} peut se justifier
				pour éviter toute ambigüité avec une autre longueur.
			\item
				\lda{long} peut être admis
				si le contexte permet de comprendre immédiatement
				l’abréviation.
			\item
				\lda{l} et \lda{lg} sont à proscrire car pas assez explicites.
			\item
				\lda{laLongueurDuRectangle} est inutilement long. 
			\item
				\lda{bidule} ou \lda{temp} ne sont pas de bons choix
				car ils n'ont aucun lien avec la donnée.
			\end{itemize}
		\end{Emphase}
		
		Nous allons également donner un \textbf{nom à l'algorithme}
		de résolution du problème.
		Cela permettra d'y faire référence dans les explications
		mais également de l'utiliser dans d'autres algorithmes.
		Généralement, un nom d'algorithme est :
		\begin{itemize}
		\item soit un verbe indiquant ce que fait l'algorithme ;
		\item soit un nom indiquant le résultat fourni.	
		\end{itemize}
	
		\begin{Emphase}
			\paragraph{Exemple.}	
			Quel nom choisir pour l'algorithme 
			qui calcule la surface d'un rectangle ?
	
			On peut envisager 
			le verbe \lda{calculerSurfaceRectangle}
			ou le nom \lda{surfaceRectangle} (notre préféré).
			On pourrait aussi simplifier en \lda{surface}
			s'il est évident qu'on traite des rectangles.
		\end{Emphase}
		
		Notons que les langages de programmation 
		imposent certaines limitations 
		(parfois différentes d’un langage à l’autre)
		ce qui peut nécessiter une modification du nom 
		lors de la traduction de l'algorithme en un programme.
	
	%----------------------------------------------
	\section{Les types}\index{types}
	%----------------------------------------------
		
		Nous allons également attribuer un \textbf{type} à chaque donnée
		ainsi qu'au résultat.
		Le \textbf{type} décrit la nature de son contenu,
		quelles valeurs elle peut prendre.
		
		Dans un premier temps, les seuls \textbf{types} autorisés 
		sont les suivants~:
		
		\begin{center}
		\begin{tabular}[t]{p{1.1cm}|p{12cm}}
			\raggedleft \lda{entier} & pour les nombres entiers\\
			\raggedleft \lda{réel} & pour les nombres réels\\
	%				\raggedleft \lda{caractère} & pour les différentes lettres et caractères 
	%						(par exemple ceux qui apparaissent sur un clavier~: 
	%						\lda{‘a’}, \lda{‘1’}, \lda{‘\#’}, etc.)
	%						\\
			\raggedleft \lda{chaine} & pour les chaines de caractères,
					les textes.
					(par exemple~: 
					\lda{"Bonjour"}, \lda{"Bonjour le monde !"}, 
					\lda{"a"}, \lda{""}\dots)
					\\
			\raggedleft \lda{booléen} & quand la valeur 
					ne peut être que \lda{vrai} ou \lda{faux}\\
		\end{tabular}
		\end{center}
	
		\begin{Emphase}
			\paragraph{Exemples.}	
			\begin{itemize}
			\item Pour la longueur, la largeur et la surface d'un rectangle, on prendra un réel.
			\item Pour le nom d'une personne, on choisira la chaine.
			\item Pour l'âge d'une personne, un entier est indiqué.
			\item Pour décrire si un étudiant est doubleur ou pas, un booléen est adapté.
			\item Pour représenter un mois, on préférera souvent un entier
				donnant le numéro du mois (par ex: 3 pour le mois de mars)
				plutôt qu'une chaine (par ex: "mars")
				car les manipulations, les calculs seront plus simples.
			\end{itemize}
		\end{Emphase}
	
		\subsection{Il n'y a pas d'unité}
		%-----------------------------------------
	
			Un type numérique indique que les valeurs possibles seront
			des nombres. Il n'y a là aucune notion d'unité.
			Ainsi, la longueur d'un rectangle, un réel, 
			peut valoir $2.5$ mais certainement pas $2.5 cm$.
			Si cette unité a de l'importance,
			il faut la spécifier dans le nom de la donnée ou en commentaire.
			
			\begin{Emphase}
				\paragraph{Exemple.}
				Faut-il préciser les unités 
				pour les dimensions d'un rectangle ?
				
				Si la longueur d'un rectangle vaut $6$, 
				on ne peut pas dire s'il s'agit de centimètres, 
				de mètres ou encore de kilomètres.
				Pour notre problème de calcul de la surface,
				ce n'est pas important ;
				la surface n'aura pas d'unité non plus.
				
				Si, par contre, 
				il est important de préciser que la longueur
				est donnée en centimètres,
				on pourrait l'expliciter en la nommant
				\lda{longueurCM}.	
			\end{Emphase}
	
		\subsection{Préciser les valeurs possibles}
		%-----------------------------------------
		
			Le type choisi n'est pas toujours assez précis.
			Souvent, la donnée ne pourra prendre que certaines valeurs.
			
			\begin{Emphase}
				\paragraph{Exemples.}	
				\begin{itemize} 
				\item Un âge est un entier qui ne peut pas être négatif.
				\item Un mois est un entier compris entre 1 et 12.
				\end{itemize}
			\end{Emphase}
			
			Ces précisions pourront être données en commentaire
			pour aider à mieux comprendre le problème et sa solution.
	
		\subsection{Le type des données complexes}
		%-----------------------------------------
		
			Parfois, aucun des types disponibles ne permet de représenter 
			la donnée.
			Il faut alors la décomposer.
			
			\begin{Emphase}
				\paragraph{Exemple.}	
				Quel type choisir 
				pour la date de naissance d'une personne ?
				
				On pourrait la représenter dans une chaine 
				(par ex: "17/3/1985")
				mais cela rendrait difficile le traitement, les calculs
				(par exemple, déterminer le numéro du mois).
				Le mieux est sans doute de la décomposer en trois parties : 
				le jour, le mois et l'année, tous des entiers.
			\end{Emphase}
			
			Plus loin dans le cours,
			nous verrons qu'il est possible de définir de nouveaux
			types de données grâce aux \emph{structures}%
			\footnote{
				L'orienté objet que vous verrez en Java
				le permet également et même mieux.%
			}. 
			On pourra alors définir et utiliser un type \lda{Date}
			et il ne sera plus nécessaire de décomposer une date en trois
			morceaux.
	
		\subsection{Exercice}
		%----------------------------------------------
		
			Quel(s) type(s) de données utiliseriez-vous pour représenter 
			\begin{itemize}
				\item le prix d’un produit en grande surface~?
				\item la taille de l'écran de votre ordinateur~?
				\item votre nom~?
				\item votre adresse~?
				\item le pourcentage de remise proposé pour un produit~?
				\item une date du calendrier~?
				\item un moment dans la journée~?
			\end{itemize}
			
	%----------------------------------------------
	\section{Résumé graphique}
	%----------------------------------------------
	
		Toutes les informations déjà collectées sur le problème
		peuvent être représentés graphiquement.
	
		\begin{Emphase}
			\paragraph{Exemple.}
			Pour le problème, de la surface du rectangle, 
			on fera le schéma suivant :
			
			\center\flowalgodd{longueur (réel)}{largeur (réel)}{surfaceRectangle}{réel}	
		\end{Emphase}
		
	%----------------------------------------------
	\section{Exemples numériques}
	%----------------------------------------------
	
		Une dernière étape pour vérifier que le problème
		est bien compris est de donner quelques exemples numériques.
		On peut les spécifier en français, 
		via un graphique ou via une notation compacte
		que nous allons présenter.
	
		\begin{Emphase}
			\paragraph{Exemples.}
			Voici différentes façons de présenter des exemples numériques
			pour le problème de calcul de la surface d'un rectangle :
			\begin{itemize}
			\item En français : 
				si la longueur du rectangle vaut 3 et sa largeur vaut 2, 
				alors sa surface vaut 6.			
			\item Via un schéma :	
				\begin{center}
					\flowalgodd{longueur (3)}{largeur (2)}{surfaceRectangle}{6}
				\end{center}
			\item En notation compacte :
				\lda{surfaceRectangle(3, 2)} donne $6$.
			\end{itemize}
		\end{Emphase}
	
	%----------------------------------------------
	\section{Exercices}
	%----------------------------------------------
	
		Pour les exercices suivants, 
		nous vous demandons d’imiter la démarche décrite dans ce chapitre, 
		à savoir :
		\begin{itemize}
			\item Déterminer quelles sont les données ;
				leur donner un nom et un type.
			\item Déterminer quel est le type du résultat.
			\item Déterminer un nom pertinent pour l'algorithme.
			\item Fournir un résumé graphique.
			\item Donner des exemples.
		\end{itemize}
	
		\begin{Exercice}{Somme de 2 nombres}
			Calculer la somme de deux nombres donnés.
			\paragraph{Solution.}%
			\footnote{%
				Nous allons de temps en temps 
				fournir des solutions.
				En algorithmique,
				il y a souvent \textbf{plus qu'une} solution possible.
				Ce n'est donc pas parce que vous avez trouvé autre chose
				que c'est mauvais.
				Mais il peut y avoir des solutions \textbf{meilleures}
				que d'autres; 
				n'hésitez jamais à montrer la vôtre
				à votre professeur pour avoir son avis.
			}
			Il y a ici clairement 2 données.
			Comme elles n'ont pas de rôle précis,
			on peut les appeler simplement \lda{nombre1}
			et \lda{nombre2}
			(\lda{nb1} et \lda{nb2} sont aussi de bons choix).
			L'énoncé ne dit pas si les nombres sont entiers ou pas;
			restons le plus général possible en prenant des réels.
			Le résultat sera de même type que les données.
			Le nom de l'algorithme pourrait être simplement \lda{somme}.
			Ce qui donne :
			\begin{center}
				\flowalgodd{nombre1 (réel)}{nombre2 (réel)}{somme}{réel}
			\end{center}			 
			Et voici quelques exemples numériques :
			
			\begin{minipage}{6cm}
				\begin{itemize}
				\item \lda{somme(3, 2)} donne $5$.
				\item \lda{somme(-3, 2)} donne $-1$.
				\end{itemize}
			\end{minipage}
			\begin{minipage}{6cm}
				\begin{itemize}
				\item \lda{somme(3, 2.5)} donne $5.5$.
				\item \lda{somme(-2.5, 2.5)} donne $0$.
				\end{itemize}
			\end{minipage}
			
		\end{Exercice}
	
		\begin{Exercice}{Moyenne de 2 nombres}
			Calculer la moyenne de deux nombres donnés.
		\end{Exercice}
		
		\begin{Exercice}{Surface d’un triangle}
			Calculer la surface d’un triangle connaissant sa base et sa hauteur.
		\end{Exercice}
	
		\begin{Exercice}{Périmètre d’un cercle}
			Calculer le périmètre d’un cercle dont on donne le rayon. 
		\end{Exercice}
	
		\begin{Exercice}{Surface d’un cercle}
			Calculer la surface d’un cercle dont on donne le rayon. 
		\end{Exercice}
	
		\begin{Exercice}{TVA}
			Si on donne un prix hors TVA, il faut lui ajouter 21\% 
			pour obtenir le prix TTC. Écrire un algorithme qui permet 
			de passer du prix HTVA au prix TTC.
		\end{Exercice}
	
		\begin{Exercice}{Les intérêts}
			Calculer les intérêts reçus après 1 an pour un montant placé en 
			banque à du 2\% d’intérêt.
		\end{Exercice}
	
		\begin{Exercice}{Placement}
			Étant donné le montant d’un capital placé (en \texteuro) 
			et le taux d’intérêt annuel (en \%), 
			calculer la nouvelle valeur de ce capital après un an.
		\end{Exercice}
	
		\begin{Exercice}{Prix TTC}
			Étant donné le prix unitaire d’un produit
			(hors TVA), le taux de TVA (en \%) et la quantité de produit vendue à
			un client, calculer le prix total à payer par ce client.
		\end{Exercice}
	
		\begin{Exercice}{Durée de trajet}
			Étant donné la vitesse moyenne en \textbf{m/s}
			d’un véhicule et la distance parcourue en \textbf{km} par ce véhicule,
			calculer la durée en secondes du trajet de ce véhicule.
		\end{Exercice}
	
		\begin{Exercice}{Allure et vitesse}
			L'allure d'un coureur est le temps qu'il met pour parcourir 1 km
			(par exemple, $4'37''$).
			On voudrait calculer sa vitesse (en km/h) à partir de son allure.
			Par exemple, la vitesse d'un coureur ayant une allure de
			$4'37''$ est de $14$ km/h. 
		\end{Exercice}
	
		\begin{Exercice}{Cote moyenne}
			Étant donné les résultats (cote entière sur
			20) de trois examens passés par un étudiant (exprimés par six nombres,
			à savoir, la cote et la pondération de chaque examen), calculer 
			la moyenne globale exprimée en pourcentage.
		\end{Exercice}
	
		\begin{Exercice}{Somme des chiffres}
			Calculer la somme des chiffres
			d’un nombre entier de 3 chiffres.
		\end{Exercice}
	
		\begin{Exercice}{Conversion HMS en secondes}
			Étant donné un moment dans la journée donné
			par trois nombres, à savoir, heure, minute et seconde, calculer le
			nombre de secondes écoulées depuis minuit.
		\end{Exercice}
	
		\begin{Exercice}{Conversion secondes en heures}
			Étant donné un temps écoulé depuis minuit.
			Si on devait exprimer ce temps sous la forme
			habituelle (heure, minute et seconde),
			que vaudrait la partie "heure".
	
			Ex~:~10000 secondes donnera 2 heures.
		\end{Exercice}
	
		\begin{Exercice}{Conversion secondes en minutes}
			Étant donné un temps écoulé depuis minuit.
			Si on devait exprimer ce temps sous la forme
			habituelle (heure, minute et seconde),
			que vaudrait la partie "minute".
	
			Ex~:~10000 secondes donnera 46 minutes.
		\end{Exercice}
	
		\begin{Exercice}{Conversion secondes en secondes}
			Étant donné un temps écoulé depuis minuit.
			Si on devait exprimer ce temps sous la forme
			habituelle (heure, minute et seconde),
			que vaudrait la partie "seconde".
	
			Ex~:~10000 secondes donnera 40 secondes.
		\end{Exercice}	
		
	
