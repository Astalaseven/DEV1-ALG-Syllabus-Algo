% ==================================
\chapter{Les variables structurées}
% ==================================

	% ==========================
	\section{Le type structuré}
	\index{structure}
	% ==========================
	
		Comme un tableau,
		une structure permet de stocker plusieurs valeurs.
		Mais, à l'inverse des tableaux,
		chaque valeur peut être de type différent
		et est désignée par un nom et pas par un numéro.
		
		Cela permet de créer un nouveau type
		permettant de représenter des données
		qui ne peuvent s'inscrire 
		dans les types simples déjà vus.
		Par exemple :
		\begin{itemize}
		\item
			Une \textbf{date} est composée de trois éléments (le jour, le mois,
			l’année). Le mois peut s'exprimer par un entier (15/10/2014) 
			ou par une chaine (15~octobre~2014).
		\item
			Un \textbf{moment} de la journée est un triple d’entiers 
			(heures, minutes, secondes).
		\item
			La localisation d’un \textbf{point} dans un plan 
			nécessite la connaissance de deux coordonnées cartésiennes 
			(l’abscisse \textit{x} et l’ordonnée \textit{y}) 
			ou polaires 
			(le rayon \textit{r} et l’angle \textit{$\theta$}).
		\item
			Une \textbf{adresse} est composée de plusieurs données~:~
			un nom de rue, 
			un numéro de maison, 
			parfois un numéro de boite postale, 
			un code postal, 
			le nom de la localité, 
			un nom ou code de pays pour une adresse à l’étranger\dots
		\end{itemize}
	
		Pour stocker et manipuler de tels ensembles de données, 
		nous utiliserons des \textbf{types structurés} 
		ou \textbf{structures}. 
		Une \textbf{structure} est donc un ensemble fini d’éléments 
		pouvant être de types distincts. 
		Chaque élément de cet ensemble, 
		appelé \textbf{champ} de la structure, possède un nom unique.
		
		Notez qu’un champ d’une structure peut lui-même être une structure. 
		Par exemple, une \textbf{carte d’identité} inclut 
		parmi ses informations une \textbf{date} de naissance, 
		l’\textbf{adresse} de son propriétaire 
		(cachée dans la puce électronique~!)\dots
	
	% ===================================
	\section{Définition d’une structure}
	% ===================================
	
		\marginicon{definition}
		La définition d’un type structuré adoptera le modèle suivant~:
	
		\begin{LDA}
		\Struct{NomDeLaStructure}
			\Decl{nomChamp1}{type1}
			\Decl{nomChamp2}{type2}
			\Stmt \dots
			\Decl{nomChampN}{typeN}
		\EndStruct
		\end{LDA}
	
		\lda{nomChamp1}, \dots, \lda{nomChampN} 
		sont les noms des différents champs de la structure, 
		et \lda{type1}, \dots, \lda{typeN} 
		les types de ces champs. 
		Tous les types peuvent être envisagés :
		les types «~simples~» (entier, réel, booléen, chaine),
		les tableaux et même d’autres types structurés 
		dont la structure aura été préalablement définie.
	
		Pour exemple, nous définissons ci-dessous trois
		types structurés que nous utiliserons souvent par la suite~:
	
		\begin{minipage}{4.5cm}
		\begin{LDA}
		\Struct{Date}
			\Decl{jour}{entier}
			\Decl{mois}{entier}
			\Decl{année}{entier}
		\EndStruct
		\end{LDA}
		\end{minipage}
	\ 
		\begin{minipage}{4.5cm}
		\begin{LDA}
		\Struct{Moment}
			\Decl{heure}{entier}
			\Decl{minute}{entier}
			\Decl{seconde}{entier}
		\EndStruct
		\end{LDA}
		\end{minipage}
	\ 
		\begin{minipage}{4.5cm}
		\begin{LDA}
		\Struct{Point}
			\Decl{x}{réel}
			\Decl{y}{réel}
		\EndStruct{}
		\Empty
		\end{LDA}
		\end{minipage}
	
	% =====================================================
	\section{Déclaration d’une variable de type structuré}
	% =====================================================
	
		Une fois un type structuré défini, 
		la déclaration de variables de ce type 
		est identique à celle des variables simples. 
		Par exemple~:
	
		\begin{LDA}
		\Decl{anniversaire, jourJ}{Date}
		\Decl{départ, arrivée, unMoment}{Moment}
		\Decl{a, b, centreGravité}{Point}
		\end{LDA}
	
	% ====================================================
	\section{Utilisation des variables de type structuré}
	% ====================================================
	
		En général, 
		pour manipuler une variable structurée ou en modifier le contenu, 
		il faut agir au niveau de ses champs en utilisant 
		les opérations permises selon leur type. 
		Pour accéder à l’un des champs d’une variable structurée, 
		il faut mentionner le nom de ce champ 
		ainsi que celui de la variable dont il fait partie.
		Nous utiliserons la notation «~pointée~»~:
	
		\begin{LDA}
		\Stmt nomVariable.nomChamp
		\end{LDA}
	
		Exemple d’instructions utilisant les variables
		déclarées au paragraphe précédent~:
	
		\begin{LDA}
		\Let anniversaire.jour \Gets 15
		\Let anniversaire.mois \Gets 10
		\Let anniversaire.année \Gets 2014
		\Let arrivée.heure \Gets départ.heure + 2
		\Let centreGravité.x \Gets (a.x + b.x) / 2
		\end{LDA}
	
		On peut cependant, dans certains cas, 
		utiliser une variable structurée de manière globale 
		(c’est-à-dire d’une façon qui agit simultanément sur chacun de ses champs). 
		Le cas le plus courant est l’affectation interne 
		entre deux variables structurées de même type, 
		par exemple~:
	
		\begin{LDA}
		\Let arrivée \Gets départ
		\end{LDA}
	
		qui résume les trois instructions suivantes~:
	
		\begin{LDA}
		\Let arrivée.heure \Gets départ.heure
		\Let arrivée.minute \Gets départ.minute
		\Let arrivée.seconde \Gets départ.seconde
		\end{LDA}
	
		Une variable structurée peut aussi 
		être le paramètre ou la valeur de retour d’un algorihtme. 
		Par exemple, l’entête d’un algorihtme 
		renvoyant le nombre de secondes 
		écoulées entre une heure de départ et d’arrivée serait~:
	
		\begin{LDA}
		\Entete{nbSecondesEcoulées}{\Par{départ\In, arrivée\In}{Moment}}{entier}
		\end{LDA}
	
		On pourra aussi lire ou afficher une variable structurée%
		\footnote{%
			Bien que, dans certains langages, 
			ces opérations devront être décomposées en une lecture ou
			écriture de chaque champ de la structure.
		}.
	
		\begin{LDA}
		\Read unMoment
		\Write unMoment
		\end{LDA}
	
		Par contre, il n’est pas autorisé d’utiliser
		les opérateurs de comparaison ({\textless}, {\textgreater}) 
		pour comparer des variables structurées (même de même type), 
		car une relation d’ordre n’accompagne pas toujours les structures utilisées. 
		En effet, s’il est naturel de vouloir comparer des dates ou des moments,
		comment définir une relation d’ordre avec les points du plan ou avec
		des cartes d’identités~?
	
		Si le besoin de comparer des variables structurées se fait sentir, 
		il faudra dans ce cas écrire 
		des modules de comparaison adaptés aux structures utilisées.
	
		Par facilité d’écriture, 
		on peut assigner tous les champs en une fois via des «~\{\}~». 
		Exemple~:
	
		\begin{LDA}
		\Let anniversaire \Gets \{1, 9, 1989\}
		\end{LDA}
	
	% =============================
	\section{Exemple d’algorithme}
	% =============================
	
		Le module ci-dessous reçoit en paramètre deux dates ; 
		la valeur renvoyée est 
		–1 si la première date est antérieure à la deuxième, 
		0 si les deux dates sont identiques 
		ou 1 si la première date est postérieure à la deuxième.
	
		\begin{LDA}
		\LComment Reçoit 2 dates en paramètres et retourne une valeur
		\LComment négative si la première date est antérieure à la deuxième
		\LComment nulle si les deux	dates sont identiques
		\LComment positive si la première date est postérieure à la deuxième.
		\Algo{comparerDates}{date1\In, date2\In : Date}{entier}
			\If{date1.année $\neq$ date2.année}
				\Return date1.année - date2.année
			\ElsIf{date1.mois $\neq$ date2.mois}
				\Return date1.mois - date2.mois 
			\Else
				\Return date1.jour - date2.jour
			\EndIf
		\EndAlgo
		\end{LDA}
	
	% =====================================
	\section{Exercices sur les structures}
	% =====================================
	
		\begin{Exercice}{Conversion moment-secondes}
			Écrire un module qui reçoit en paramètre un
			moment d’une journée et qui retourne le nombre de secondes écoulées
			depuis minuit jusqu’à ce moment.
		\end{Exercice}
		
		\begin{Exercice}{Conversion secondes-moment}
			Écrire un module qui reçoit en paramètre un
			nombre de secondes écoulées dans une journée à partir de minuit et qui
			retourne le moment correspondant de la journée.
		\end{Exercice}
		
		\begin{Exercice}{Temps écoulé entre 2 moments}
			Écrire un module qui reçoit en paramètres deux
			moments d’une journée et qui retourne le nombre de secondes séparant
			ces deux moments.
		\end{Exercice}
		
		\begin{Exercice}{Milieu de 2 points}
			Écrire un module recevant les coordonnées de
			deux points distincts du plan et qui retourne les coordonnées du point
			situé au milieu des deux.
		\end{Exercice}
		
		\begin{Exercice}{Distance entre 2 points}
			Écrire un module recevant les coordonnées de
			deux points distincts du plan et qui retourne
			la distance entre ces deux points.
		\end{Exercice}
		
		\begin{Exercice}{Un cercle}
			Définir un type \textbf{Cercle} pouvant décrire de façon
			commode un cercle quelconque dans un espace à deux dimensions. 	
			Écrire ensuite
			
			\begin{enumerate}[label=\alph*)]
			\item {
				un module calculant la surface du cercle reçu en paramètre ;}
			\item {
				un module recevant 2 points en paramètre et retournant le cercle dont le
				diamètre est le segment reliant ces 2 points ;}
			\item {
				un module qui détermine si un point donné est dans un cercle ;}
			\item {
				un module qui indique si 2 cercles ont une intersection.
			}
			\end{enumerate}
		\end{Exercice}
		
		
		\begin{Exercice}{Un rectangle}
			\marginicon{java}
			Définir un type \textbf{Rectangle} pouvant décrire de façon
			commode un rectangle dans un espace à deux dimensions et dont les côtés
			sont parallèles aux axes des coordonnées. 	
			Écrire ensuite
		
			\begin{enumerate}[label=\alph*)]
			\item {
				un module calculant le périmètre d’un rectangle reçu en paramètre ;}
			\item {
				un module calculant la surface d’un rectangle reçu en paramètre ;}
			\item {
				un module recevant en paramètre un rectangle R et les coordonnées
				d’un point P, et renvoyant 
				\textbf{vrai} si et seulement si le point P est à
				l’intérieur du rectangle R ;}
			\item {
				un module recevant en paramètre un rectangle R et les coordonnées
				d’un point P, et renvoyant 
				\textbf{vrai} si et seulement si le point P est sur le bord du
				rectangle R ;}
			\item {
				un module recevant en paramètre deux rectangles et renvoyant la valeur
				booléenne \textbf{vrai} si et seulement si ces deux rectangles ont une
				intersection.}
			\end{enumerate}
		\end{Exercice}
		
		\begin{Exercice}{Validation de date}
			Écrire un algorithme qui valide une date reçue en paramètre 
			sous forme d’une structure.
		\end{Exercice}

	
