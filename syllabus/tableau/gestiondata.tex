%==========================================
\chapter{Gérer les données dans un tableau}
%==========================================

	Une utilisation courante des tableaux
	est le stockage des données changeantes.
	Lors de l'évolution de l'algorithme
	des données sont ajoutées, recherchées, supprimées, modifiées.
	
	Par exemple, imaginons que l'école 
	organise un événement libre et gratuit pour les étudiants.
	Mais pour y assister, ils doivent s'inscrire.
	On veut fournir ce qu'il faut pour :
	\begin{itemize}
	\item inscrire un éudiant ;
	\item vérifier si un étudiant est inscrit ;
	\item désinscrire un étudiant (s'il change d'avis par exemple) ;
	\item afficher la liste des inscrits.
	\end{itemize}

	Pour ce faire,
	on pourra stocker les numéros des étudiants inscrits
	dans un tableau%
	\footnote{%
		Une vraie solution de ce problème
		utiliserait probablement une base de données
		mais ça c'est une autre histoire et un autre cours ;)
	}.
	
	On va envisager deux cas :
	\begin{enumerate}
	\item
		Les numéros d'étudiants seront placés dans le tableau sans ordre imposé.
	\item
		Les numéros d'étudiants seront placés dans l'ordre.		
	\end{enumerate}
	
	% ==============================================
	\section{Données non triées} 
	% ==============================================
		
		Pour garder les numéros d'étudiants dans un tableau,
		il nous faut gérer sa taille logique.
		On a donc deux variables :
		\begin{itemize}
		\item \lda{inscrits} : le tableau des numéros d'étudiants
		\item \lda{nbInscrits} : le nombre d'étudiants actuellement inscrits (la taille logique du tableau)
		\end{itemize}
				
		Par exemple, on pourrait avoir la situation suivante :
		\begin{center}
			\lda{inscrits} = 
			\smallskip
			\begin{tabular}{*{10}{|>{\centering\arraybackslash}m{1cm}}|}
				\hline
				42010 & 41390 & 43342 & 42424 & ? & ? & ? & ? & \dots \\
				\hline
			\end{tabular}
			\smallskip
			\lda{nbInscrits = 4}
		\end{center}
		indiquant qu'il y a pour les moment quatre étudiants inscrits
		dont les numéros sont ceux repris 
		dans les quatre premières cases du tableau.
		
		\subsection{Inscription}
		
			Commençons par l'inscription.
			Pour inscrire un étudiant, il suffit de l'ajouter à la suite
			dans le tableau.
			Par exemple, en partant de la situation décrite ci-dessus,
			l'inscription de l'étudiant numéro 42123 aboutit à la situation
			suivante :
			\begin{center}
				\lda{inscrits} = 
				\smallskip
				\begin{tabular}{*{10}{|>{\centering\arraybackslash}m{1cm}}|}
					\hline
					42010 & 41390 & 43342 & 42424 & 42123 & ? & ? & ? & \dots \\
					\hline
				\end{tabular}
				\smallskip
				\lda{nbInscrits = 5}
			\end{center}
			
			L'algorithme est assez simple :
			
			\begin{LDA}
				\LComment{Inscrit un étudiant de numéro donné.}
				\Algo{inscrire}{\Par{inscrits\In\Out}{\Array{n}{entiers}}, \Par{nbInscrits\In\Out}{entier}, \Par{numéro}{entier}}{}
					\LComment{On peut imaginer vérifier que l'étudiant n'est pas déjà inscrit}
					\LComment{On peut imaginer vérifier qu'il reste de la place (c-à-d que le tableau n'est pas plein)}
					\Let inscrits[nbInscrits] \Gets numéro
					\Let nbInscrits \Gets nbInscrits + 1
				\EndAlgo
			\end{LDA}
		
		\subsection{Vérifier inscription}
	
			Pour vérifier si un étudiant est déja inscrit,
			il faut le rechercher dans la partie utilisée du tableau.
			Cette recherche se fait via un parcours avec sortie
			prématurée. On pourrait se contenter de retourner un
			booléen indiquant si oui ou non on l'a trouvé
			mais retournons plutôt un entier indiquant l'indice où
			il est (-1 si on ne l'a pas trouvé), ce sera plus utile.
			
			\begin{LDA}
				\LComment{Vérifie si un étudiant est inscrit et donne sa position (-1 si non inscrit)}
				\Algo{vérifier}{\Par{inscrits\In}{\Array{n}{entiers}}, \Par{nbInscrits\In}{entier}, \Par{numéro}{entier}}{entier}
					\Decl{i}{entier}
					\Let i \Gets 0
					\While{i < nbInscrits ET inscrits[i] $\ne$ numéro}
						\Let i \Gets i + 1
					\EndWhile
					\If{i < nbInscrits}
						\Return i
					\Else
						\Return -1
					\EndIf
				\EndAlgo
			\end{LDA}
			
			La fiche (A ECRIRE) reprend cet algorithme
			dans un cadre plus général et met en avant
			les variantes et les pièges à éviter.

		\subsection{Désinscription}
		
			Pour désinscrire un étudiant,
			il faut le trouver dans le tableau et l'enlever.
			Attention, il ne peut pas y avoir de trou dans un tableau
			(il n'y a pas de case vide).
			Le plus simple est d'y déplacer le dernier élément.
			Par exemple,
			reprenons la situation après inscription de l'étudiant
			numéro 42123.
			La désincription de l'étudiant numéro 41390 donnerait%
			\footnote{
				Nous avons volontairement indiqué le 42123 en 5\ieme{} position.
				Il est toujours là mais ne sera pas considéré par les
				algorithmes puisque cette case est au-delà de la taille logique
				(donnée par la variable \lda{nbInscrits}).
			}

			\begin{center}
				\lda{inscrits} = 
				\smallskip
				\begin{tabular}{*{10}{|>{\centering\arraybackslash}m{1cm}}|}
					\hline
					42010 & 42123 & 43342 & 42424 & 42123 & ? & ? & ? & \dots \\
					\hline
				\end{tabular}
				\smallskip
				\lda{nbInscrits = 4}
			\end{center}
			
			L'algorithme est assez simple à écrire
			si on utilise la recherche écrite juste avant :
			
			\begin{LDA}
				\LComment{Vérifie si un étudiant est inscrit et donne sa position (-1 si non inscrit)}
				\Algo{désinscrire}{\Par{inscrits\In\Out}{\Array{n}{entiers}}, \Par{nbInscrits\In\Out}{entier}, \Par{numéro}{entier}}{entier}
					\Decl{pos}{entier}
					\Let pos \Gets vérifier(inscrits, nbInscrits, numéro)
					\LComment{On pourrait vérifier et générer une erreur si l'étudiant n'est pas inscrit}
					\Let inscrits[pos] \Gets inscrits[nbInscrits-1]
					\Let nbInscrits \Gets nbInscrits - 1					
				\EndAlgo
			\end{LDA}
			
		\subsection{Exercices}
			
			\begin{Exercice}{Éviter la double inscription}
				Comment modifier l'algorithme d'inscription
				pour s'assurer qu'un étudiant ne s'inscrive pas deux fois ?
			\end{Exercice}

			\begin{Exercice}{Limite au nombre d'inscriptions}
				Comment modifier l'algorithme d'inscription
				pour refuser une inscription si le nombre maximal
				de participant est atteint
				en supposant que ce maximum est égal à la taille physique du tableau ?
			\end{Exercice}

			\begin{Exercice}{Vérifier la déinscription}
				Que se passerait-il avec l'algorithme
				de désinscription tel qu'il est
				si on demande à désinscrire un étudiant non inscrit ?
				Que suggérez-vous comme changement ?
			\end{Exercice}

			\begin{Exercice}{Optimiser la déinscription}
				Dans l'algorithme de désinscription tel qu'il est,
				voyez-vous un cas où on effectue une opération inutile ?
				Avez-vous une proposition pour l'éviter ?
			\end{Exercice}

	% ==============================================
	\section{Données triées} 
	% ==============================================

	% ==============================================
	\section{La recherche dichotomique} 
	% ==============================================

	% ==============================================
	\section{Introduction à la complexité} 
	% ==============================================
