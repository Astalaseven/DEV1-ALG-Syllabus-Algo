%===============
\chapter{Le tri}
%===============

	\marginicon{objectif}
	Dans ce chapitre nous voyons quelques algorithmes simples pour trier un
	ensemble d'informations~: recherche des maxima, tri
	par insertion et tri bulle dans un tableau. 
	Des algorithmes plus efficaces seront vus
	en deuxième année.

%===================
\section{Motivation}
%===================

	Dans le chapitre précédent,
	nous avons vu que la recherche d'information est beaucoup
	plus rapide dans un tableau trié 
	grâce à l'algorithme de recherche dichotomique.
	Nous avons vu comment \textbf{garder} un ensemble trié
	en insérant toute nouvelle valeur au \emph{bon} endroit.

	Dans certaines situations,
	on est amené à devoir trié tout un tableau.

	\begin{enumerate}
		\item 
			D’abord les situations impliquant le classement total d’un ensemble de
			données «~bru-tes~», c’est-à-dire complètement désordonnées. Prenons
			pour exemple les feuilles récoltées en vrac à l’issue d’un examen ; il
			y a peu de chances que celles-ci soient remises à l’examinateur de
			manière ordonnée ; celui-ci devra donc procéder au tri de l’ensemble
			des copies, par exemple par ordre alphabétique des noms des étudiants,
			ou par numéro de groupe etc.
		\item 
			Enfin, les situations qui consistent à devoir re-trier des données
			préalablement ordonnées sur un autre critère. Prenons l’exemple d’un
			paquet de copies d’examen déjà triées sur l’ordre alphabétique des noms
			des étudiants, et qu’on veut re-trier cette fois-ci sur les numéros de
			groupe. Il est clair qu’une méthode efficace veillera à conserver
			l’ordre alphabétique déjà présent dans la première situation afin que
			les copies apparaissent dans cet ordre dans chacun des groupes.
	\end{enumerate}
	
	Le dernier cas illustre un classement sur une \textbf{clé complexe}
	(ou \textbf{composée}) impliquant la comparaison de plusieurs champs
	d’une même structure~: le premier classement se fait sur le numéro de
	groupe, et à numéro de groupe égal, l’ordre se départage sur le nom de
	l’étudiant. On dira de cet ensemble qu’il est classé en \textbf{majeur}
	sur le numéro de groupe et en \textbf{mineur} sur le nom d’étudiant.

	Notons que certains tris sont dits \textbf{stables} parce
	qu'en cas de tri sur une nouvelle clé, l’ordre de la
	clé précédente est préservé pour des valeurs identiques de la nouvelle
	clé, ce qui évite de faire des comparaisons sur les deux champs à la
	fois. Les méthodes nommées \textbf{tri par insertion}, \textbf{tri
	bulle} et \textbf{tri par recherche de minima successifs }(que nous
	allons aborder dans ce chapitre) sont stables.

	Certains tris sont évidemment plus performants que d’autres. Le choix
	d’un tri particulier dépend de la taille de l’ensemble à trier et de la
	manière dont il se présente, c’est-à-dire déjà plus ou moins ordonné.
	La performance quant à elle se juge sur deux critères~: le nombre de
	tests effectués (comparaisons de valeurs) et le nombre de transferts de
	valeurs réalisés. 
	
	Les algorithmes classiques de tri présentés dans ce chapitre le sont
	surtout à titre pédagogique. Ils ont tous une «~complexité en
	$n^2$~», ce qui veut dire que si $n$ est le nombre
	d’éléments à trier, le nombre d’opérations élémentaires (tests et
	transferts de valeurs) est proportionnel à $n^2$. Ils conviennent
	donc pour des ensembles de taille «~raisonnable~», mais peuvent devenir
	extrêmement lents à l’exécution pour le tri de grands ensembles, comme
	par exemple les données de l’annuaire téléphonique. Plusieurs solutions
	existent, comme la méthode de tri \textbf{Quicksort}. Cet algorithme
	très efficace faisant appel à la récursivité et qui sera étudié en
	deuxième année a une complexité en $n \log(n)$. 

	\marginicon{attention}
	Dans ce chapitre, les algorithmes traiteront du tri dans un
	\textbf{tableau d’entiers à une }\textbf{dimension}. Toute autre
	situation peut bien entendu se ramener à celle-ci moyennant la
	définition de la relation d’ordre propre au type de données utilisé. Ce
	sera par exemple, l’ordre alphabétique pour les chaines de caractères,
	l’ordre chronologique pour des objets \lda{Date} ou
	\lda{Moment} (que nous verrons
	plus tard), etc. De plus, le seul ordre envisagé sera l’ordre
	\textbf{croissant} des données. Plus loin, on envisagera le tri 
	d'autres structures de données.

	Enfin, dans toutes les méthodes de tri abordées, nous supposerons la
	taille physique du tableau à trier (notée $n$) égale à sa taille
	logique, celle-ci n’étant pas modifiée par l’action de tri.

%==========================
\clearpage
\section{Tri par insertion}
%===========================

	Cette méthode de tri repose sur le principe d’insertion de valeurs dans
	un tableau ordonné. 

	\paragraph{Description de l’algorithme.}
	
	Le tableau à trier sera à chaque étape subdivisé en deux sous-tableaux~:
	le premier cadré à gauche contiendra des éléments déjà ordonnés, et le
	second, cadré à droite, ceux qu’il reste à insérer dans le sous-tableau
	trié. Celui-ci verra sa taille s’accroitre au fur et à mesure des
	insertions, tandis que celle du sous-tableau des éléments non triés
	diminuera progressivement.

	Au départ de l’algorithme, le sous-tableau trié est le premier élément
	du tableau. Comme il ne possède qu’un seul élément, ce sous-tableau est
	donc bien ordonné ! Chaque étape consiste ensuite à prendre le premier
	élément du sous-tableau non trié et à l’insérer à la bonne place dans
	le sous-tableau trié.

	Prenons comme exemple un tableau \lda{tab} de 20 entiers. 	
	Au départ, le	sous-tableau trié est formé du premier élément, 
	\lda{tab[0]} qui vaut 20~:

	\begin{tabular}{*{20}{>{\centering\sffamily\itshape\arraybackslash}m{0.21cm}}}
		 \textcolor{gray}{\scriptsize 0} &
		 \textcolor{gray}{\scriptsize 1} &
		 \textcolor{gray}{\scriptsize 2} &
		 \textcolor{gray}{\scriptsize 3} &
		 \textcolor{gray}{\scriptsize 4} &
		 \textcolor{gray}{\scriptsize 5} &
		 \textcolor{gray}{\scriptsize 6} &
		 \textcolor{gray}{\scriptsize 7} &
		 \textcolor{gray}{\scriptsize 8} &
		 \textcolor{gray}{\scriptsize 9} &
		 \textcolor{gray}{\scriptsize 10} &
		 \textcolor{gray}{\scriptsize 11} &
		 \textcolor{gray}{\scriptsize 12} &
		 \textcolor{gray}{\scriptsize 13} &
		 \textcolor{gray}{\scriptsize 14} &
		 \textcolor{gray}{\scriptsize 15} &
		 \textcolor{gray}{\scriptsize 16} &
		 \textcolor{gray}{\scriptsize 17} &
		 \textcolor{gray}{\scriptsize 18} &
		 \textcolor{gray}{\scriptsize 19}
		 \\
	\end{tabular}
	\\
	\begin{tabular}{|*{20}{>{\centering\arraybackslash}m{0.20cm}|}}
		\hline
		{\cellcolor{gray!25}20} &
		{ 12} &
		{ 18} &
		{ 17} &
		{ 15} &
		{ 14} &
		{ 15} &
		{ 16} &
		{ 18} &
		{ 17} &
		{ 12} &
		{ 14} &
		{ 16} &
		{ 18} &
		{ 15} &
		{ 15} &
		{ 19} &
		{ 11} &
		{ 11} &
		{ 13}\\\hline
	\end{tabular}

	\medskip
	
	L’étape suivante consiste à insérer \lda{tab[1]}, qui vaut 12, dans ce sous-tableau, de
	taille 2~:

	\begin{tabular}{*{20}{>{\centering\sffamily\itshape\arraybackslash}m{0.21cm}}}
		 \textcolor{gray}{\scriptsize 0} &
		 \textcolor{gray}{\scriptsize 1} &
		 \textcolor{gray}{\scriptsize 2} &
		 \textcolor{gray}{\scriptsize 3} &
		 \textcolor{gray}{\scriptsize 4} &
		 \textcolor{gray}{\scriptsize 5} &
		 \textcolor{gray}{\scriptsize 6} &
		 \textcolor{gray}{\scriptsize 7} &
		 \textcolor{gray}{\scriptsize 8} &
		 \textcolor{gray}{\scriptsize 9} &
		 \textcolor{gray}{\scriptsize 10} &
		 \textcolor{gray}{\scriptsize 11} &
		 \textcolor{gray}{\scriptsize 12} &
		 \textcolor{gray}{\scriptsize 13} &
		 \textcolor{gray}{\scriptsize 14} &
		 \textcolor{gray}{\scriptsize 15} &
		 \textcolor{gray}{\scriptsize 16} &
		 \textcolor{gray}{\scriptsize 17} &
		 \textcolor{gray}{\scriptsize 18} &
		 \textcolor{gray}{\scriptsize 19}
		 \\
	\end{tabular}
	\\
	\begin{tabular}{|*{20}{>{\centering\arraybackslash}m{0.20cm}|}}
		\hline
		{\cellcolor{gray!25}12} &
		{\cellcolor{gray!25}20} &
		{18} &
		{ 17} &
		{ 15} &
		{ 14} &
		{ 15} &
		{ 16} &
		{ 18} &
		{ 17} &
		{ 12} &
		{ 14} &
		{ 16} &
		{ 18} &
		{ 15} &
		{ 15} &
		{ 19} &
		{ 11} &
		{ 11} &
		{ 13}\\\hline
	\end{tabular}

	\medskip

	Ensuite, c’est au tour de \lda{tab[2]}, qui vaut 18, d’être inséré~:

	\begin{tabular}{*{20}{>{\centering\sffamily\itshape\arraybackslash}m{0.21cm}}}
		 \textcolor{gray}{\scriptsize 0} &
		 \textcolor{gray}{\scriptsize 1} &
		 \textcolor{gray}{\scriptsize 2} &
		 \textcolor{gray}{\scriptsize 3} &
		 \textcolor{gray}{\scriptsize 4} &
		 \textcolor{gray}{\scriptsize 5} &
		 \textcolor{gray}{\scriptsize 6} &
		 \textcolor{gray}{\scriptsize 7} &
		 \textcolor{gray}{\scriptsize 8} &
		 \textcolor{gray}{\scriptsize 9} &
		 \textcolor{gray}{\scriptsize 10} &
		 \textcolor{gray}{\scriptsize 11} &
		 \textcolor{gray}{\scriptsize 12} &
		 \textcolor{gray}{\scriptsize 13} &
		 \textcolor{gray}{\scriptsize 14} &
		 \textcolor{gray}{\scriptsize 15} &
		 \textcolor{gray}{\scriptsize 16} &
		 \textcolor{gray}{\scriptsize 17} &
		 \textcolor{gray}{\scriptsize 18} &
		 \textcolor{gray}{\scriptsize 19}
		 \\
	\end{tabular}
	\\
	\begin{tabular}{|*{20}{>{\centering\arraybackslash}m{0.20cm}|}}
		\hline
		{\cellcolor{gray!25}12} &
		{\cellcolor{gray!25}18} &
		{\cellcolor{gray!25}20} &
		{ 17} &
		{ 15} &
		{ 14} &
		{ 15} &
		{ 16} &
		{ 18} &
		{ 17} &
		{ 12} &
		{ 14} &
		{ 16} &
		{ 18} &
		{ 15} &
		{ 15} &
		{ 19} &
		{ 11} &
		{ 11} &
		{ 13}\\\hline
	\end{tabular}
	
	\medskip

	Et ainsi de suite jusqu’à insertion du dernier élément dans le
	sous-tableau trié. 

	\paragraph{Algorithme.}

	On combine la recherche de la position d’insertion et le décalage
	concomitant de valeurs.

	\begin{LDA}
		\LComment Trie le tableau reçu en paramètre (via un tri par insertion).
		\Algo{triInsertion}{\Par{tab\InOut}{\Array{n}{entiers}}}{}
			\Decl{i, j, valAInsérer}{entiers}
			\For{i}{1}{n-1}
				\Let valAInsérer \Gets tab[i]
				\LComment recherche de l’endroit où insérer valAInsérer dans le 
				\LComment sous-tableau trié et décalage simultané des éléments
				\Let j \Gets i - 1
				\While{j ${\geq}$ 0 ET valAInsérer < tab[j]}
					\Let tab[j+1] \Gets tab[j]
					\Let j \Gets j – 1
				\EndWhile
				\Let tab[j+1] \Gets valAInsérer
			\EndFor
		\EndAlgo
	\end{LDA}

%================================================
\clearpage
\section{Tri par sélection des minima successifs}
%=================================================
	
	Dans ce tri, on recherche à chaque étape la plus petite valeur de
	l’ensemble non encore trié et on peut la placer immédiatement 
	à sa position	définitive.

	\paragraph{Description de l’algorithme.}

	Prenons par exemple un tableau de 20 entiers. 
	
	La première étape consiste
	à rechercher la valeur minimale du tableau. Il s’agit de l’élément
	d’indice 9 et de valeur 17.
	
	\begin{tabular}{*{20}{>{\centering\sffamily\itshape\arraybackslash}m{0.21cm}}}
		 \textcolor{gray}{\scriptsize 0} &
		 \textcolor{gray}{\scriptsize 1} &
		 \textcolor{gray}{\scriptsize 2} &
		 \textcolor{gray}{\scriptsize 3} &
		 \textcolor{gray}{\scriptsize 4} &
		 \textcolor{gray}{\scriptsize 5} &
		 \textcolor{gray}{\scriptsize 6} &
		 \textcolor{gray}{\scriptsize 7} &
		 \textcolor{gray}{\scriptsize 8} &
		 \textcolor{gray}{\scriptsize 9} &
		 \textcolor{gray}{\scriptsize 10} &
		 \textcolor{gray}{\scriptsize 11} &
		 \textcolor{gray}{\scriptsize 12} &
		 \textcolor{gray}{\scriptsize 13} &
		 \textcolor{gray}{\scriptsize 14} &
		 \textcolor{gray}{\scriptsize 15} &
		 \textcolor{gray}{\scriptsize 16} &
		 \textcolor{gray}{\scriptsize 17} &
		 \textcolor{gray}{\scriptsize 18} &
		 \textcolor{gray}{\scriptsize 19}
		 \\
	\end{tabular}
	\\
	\begin{tabular}{|*{20}{>{\centering\arraybackslash}m{0.20cm}|}}
		\hline
		{20} &
		{ 52} &
		{ 61} &
		{ 47} &
		{ 82} &
		{ 64} &
		{ 95} &
		{ 66} &
		{ 84} &
		{\cellcolor{gray!25}17} &
		{ 32} &
		{ 24} &
		{ 46} &
		{ 48} &
		{ 75} &
		{ 55} &
		{ 19} &
		{ 61} &
		{ 21} &
		{ 30}\\\hline
	\end{tabular}

	Celui-ci devrait donc apparaitre en 1\iere{} position du tableau. 
	Hors, cette position est occupée par la valeur 20. 
	Comment procéder dès lors pour ne perdre aucune valeur 
	et sans faire appel à un second tableau ? 
	La solution est simple, il suffit d’échanger 
	le contenu des deux éléments d’indices 0 et 9~:
	
	\begin{tabular}{*{20}{>{\centering\sffamily\itshape\arraybackslash}m{0.21cm}}}
		 \textcolor{gray}{\scriptsize 0} &
		 \textcolor{gray}{\scriptsize 1} &
		 \textcolor{gray}{\scriptsize 2} &
		 \textcolor{gray}{\scriptsize 3} &
		 \textcolor{gray}{\scriptsize 4} &
		 \textcolor{gray}{\scriptsize 5} &
		 \textcolor{gray}{\scriptsize 6} &
		 \textcolor{gray}{\scriptsize 7} &
		 \textcolor{gray}{\scriptsize 8} &
		 \textcolor{gray}{\scriptsize 9} &
		 \textcolor{gray}{\scriptsize 10} &
		 \textcolor{gray}{\scriptsize 11} &
		 \textcolor{gray}{\scriptsize 12} &
		 \textcolor{gray}{\scriptsize 13} &
		 \textcolor{gray}{\scriptsize 14} &
		 \textcolor{gray}{\scriptsize 15} &
		 \textcolor{gray}{\scriptsize 16} &
		 \textcolor{gray}{\scriptsize 17} &
		 \textcolor{gray}{\scriptsize 18} &
		 \textcolor{gray}{\scriptsize 19}
		 \\
	\end{tabular}
	\\
	\begin{tabular}{|*{20}{>{\centering\arraybackslash}m{0.20cm}|}}
		\hline
		{\cellcolor{gray!25}17} &
		{ 52} &
		{ 61} &
		{ 47} &
		{ 82} &
		{ 64} &
		{ 95} &
		{ 66} &
		{ 84} &
		{ 20} &
		{ 32} &
		{ 24} &
		{ 46} &
		{ 48} &
		{ 75} &
		{ 55} &
		{ 19} &
		{ 61} &
		{ 21} &
		{ 30}\\\hline
	\end{tabular}
	
	Le tableau se subdivise à présent en deux sous-tableaux, 
	un sous-tableau déjà trié 
	(pour le moment réduit au seul élément $tab[0]$) 
	et le sous-tableau des autres valeurs non encore triées 
	(de l’indice 1 à 19).
	On recommence ce processus dans ce second sous-tableau~: 
	le minimum est à présent l'élément d’indice 16 et de valeur 19.
	Celui-ci viendra donc en 2\ieme{} position, 
	échangeant sa place avec la valeur 52 qui s’y trouvait~:

	\begin{tabular}{*{20}{>{\centering\sffamily\itshape\arraybackslash}m{0.21cm}}}
		 \textcolor{gray}{\scriptsize 0} &
		 \textcolor{gray}{\scriptsize 1} &
		 \textcolor{gray}{\scriptsize 2} &
		 \textcolor{gray}{\scriptsize 3} &
		 \textcolor{gray}{\scriptsize 4} &
		 \textcolor{gray}{\scriptsize 5} &
		 \textcolor{gray}{\scriptsize 6} &
		 \textcolor{gray}{\scriptsize 7} &
		 \textcolor{gray}{\scriptsize 8} &
		 \textcolor{gray}{\scriptsize 9} &
		 \textcolor{gray}{\scriptsize 10} &
		 \textcolor{gray}{\scriptsize 11} &
		 \textcolor{gray}{\scriptsize 12} &
		 \textcolor{gray}{\scriptsize 13} &
		 \textcolor{gray}{\scriptsize 14} &
		 \textcolor{gray}{\scriptsize 15} &
		 \textcolor{gray}{\scriptsize 16} &
		 \textcolor{gray}{\scriptsize 17} &
		 \textcolor{gray}{\scriptsize 18} &
		 \textcolor{gray}{\scriptsize 19}
		 \\
	\end{tabular}
	\\
	\begin{tabular}{|*{20}{>{\centering\arraybackslash}m{0.20cm}|}}
		\hline
		{\cellcolor{gray!25}17} &
		{\cellcolor{gray!25}19} &
		{ 61} &
		{ 47} &
		{ 82} &
		{ 64} &
		{ 95} &
		{ 66} &
		{ 84} &
		{ 20} &
		{ 32} &
		{ 24} &
		{ 46} &
		{ 48} &
		{ 75} &
		{ 55} &
		{ 52} &
		{ 61} &
		{ 21} &
		{ 30}\\\hline
	\end{tabular}

	Le sous-tableau trié est maintenant formé des deux premiers éléments, 
	et le sous-tableau non trié par les 18 éléments suivants. 
	
	Et ainsi de suite, jusqu'à obtenir un sous-tableau trié 
	comprend tout le tableau.
	En pratique l’arrêt se fait une étape avant 
	car lorsque le sous-tableau non trié n’est plus que de taille 1, 
	il contient nécessairement le maximum de l’ensemble.

	\paragraph{Algorithme.}
	
	\begin{LDA}
		\LComment Trie le tableau reçu en paramètre (via un tri par sélection des minima successifs).
		\Algo{triSélectionMinimaSuccessifs}{\Par{tab\InOut}{\Array{n}{entiers}}}{}
			\Decl{i, indiceMin}{entier}
			\For{i}{0}{n – 2}
			\RComment i correspond à l’étape de l’algorithme
				\Let indiceMin \Gets positionMin( tab, i, n-1 )
				\Stmt swap( tab[i], tab[indiceMin] )
			\EndFor
		\EndAlgo
	\end{LDA}

	\begin{LDA}
		\LComment Retourne l’indice du minimum entre les indices début et fin du tableau reçu.
		\Algo{positionMin}{\Par{tab\InOut}{\Array{n}{entiers}}, Par{début, fin}{entiers}}{entier}
			\Decl{indiceMin, i}{entiers}
			\Let indiceMin \Gets début
			\For{i}{début+1}{fin}
				\If{tab[i] < tab[indiceMin]}
					\Decl indiceMin \Gets i
				\EndIf
			\EndFor
			\Return indiceMin
		\EndAlgo
	\end{LDA}

	\begin{LDA}
		\LComment {Échange le contenu de 2 variables.}
		\Algo{swap}{\Par{a\InOut, b\InOut}{entiers}}{}
			\Decl{aux}{entiers}
			\Let aux \Gets a
			\Let a \Gets b
			\Let b \Gets aux
		\EndAlgo
	\end{LDA}

%==================
\section{Tri bulle}
%===================
	
	Il s’agit d’un tri par \textbf{permutations} 
	ayant pour but d’amener à chaque étape à la «~surface~» 
	du sous-tableau non trié 
	(on entend par là l’élément d’indice minimum) 
	la valeur la plus petite, 
	appelée la \textbf{bulle}. 
	La caractéristique de cette méthode est que les
	comparaisons ne se font qu’entre éléments consécutifs du tableau.

	\paragraph{Description de l’algorithme}

	Prenons pour exemple un tableau de taille 14. En partant de la fin du
	tableau, on le parcourt vers la gauche en comparant chaque couple de
	valeurs consécutives. Quand deux valeurs sont dans le désordre, on les
	permute. Le premier parcours s’achève lorsqu’on arrive à l’élément
	d’indice 0 qui contient alors la «~bulle~»,
	c'est-à-dire la plus petite valeur du tableau, soit 1~:

	\begin{tabular}{*{14}{>{\centering\sffamily\itshape\arraybackslash}m{0.31cm}}}
		 \textcolor{gray}{\scriptsize 1} &
		 \textcolor{gray}{\scriptsize 2} &
		 \textcolor{gray}{\scriptsize 3} &
		 \textcolor{gray}{\scriptsize 4} &
		 \textcolor{gray}{\scriptsize 5} &
		 \textcolor{gray}{\scriptsize 6} &
		 \textcolor{gray}{\scriptsize 7} &
		 \textcolor{gray}{\scriptsize 8} &
		 \textcolor{gray}{\scriptsize 9} &
		 \textcolor{gray}{\scriptsize 10} &
		 \textcolor{gray}{\scriptsize 11} &
		 \textcolor{gray}{\scriptsize 12} &
		 \textcolor{gray}{\scriptsize 13} &
		 \textcolor{gray}{\scriptsize 14}
		\\
	\end{tabular}
	\\
	\begin{tabular}{|*{14}{>{\centering\arraybackslash}m{0.3cm}|}}
		\hline
		{10} &
		{  5} &
		{ 12} &
		{ 15} &
		{  4} &
		{  8} &
		{  1} &
		{  7} &
		{ 12} &
		{ 11} &
		{  3} &
		{  6} &
		{  5} &
		{\cellcolor{gray!25}4}\\\hline
	\end{tabular}
	\\
	\begin{tabular}{|*{14}{>{\centering\arraybackslash}m{0.3cm}|}}
		\hline
		{10} &
		{  5} &
		{ 12} &
		{ 15} &
		{  4} &
		{  8} &
		{  1} &
		{  7} &
		{ 12} &
		{ 11} &
		{  3} &
		{  6} &
		{\cellcolor{gray!25}4} &
		{  5}\\\hline
	\end{tabular}
	\\
	\begin{tabular}{|*{14}{>{\centering\arraybackslash}m{0.3cm}|}}
		\hline
		{10} &
		{  5} &
		{ 12} &
		{ 15} &
		{  4} &
		{  8} &
		{  1} &
		{  7} &
		{ 12} &
		{ 11} &
		{  3} &
		{\cellcolor{gray!25}4} &
		{  6} &
		{  5}\\\hline
	\end{tabular}
	\\
	\begin{tabular}{|*{14}{>{\centering\arraybackslash}m{0.3cm}|}}
		\hline
		{10} &
		{  5} &
		{ 12} &
		{ 15} &
		{  4} &
		{  8} &
		{  1} &
		{  7} &
		{ 12} &
		{ 11} &
		{\cellcolor{gray!25}3} &
		{  4} &
		{  6} &
		{  5}\\\hline
	\end{tabular}
	\\
	\begin{tabular}{|*{14}{>{\centering\arraybackslash}m{0.3cm}|}}
		\hline
		{10} &
		{  5} &
		{ 12} &
		{ 15} &
		{  4} &
		{  8} &
		{  1} &
		{  7} &
		{ 12} &
		{\cellcolor{gray!25}3} &
		{ 11} &
		{  4} &
		{  6} &
		{  5}\\\hline
	\end{tabular}
	\\
	\begin{tabular}{|*{14}{>{\centering\arraybackslash}m{0.3cm}|}}
		\hline
		{10} &
		{  5} &
		{ 12} &
		{ 15} &
		{  4} &
		{  8} &
		{  1} &
		{  7} &
		{\cellcolor{gray!25}3} &
		{ 12} &
		{ 11} &
		{  4} &
		{  6} &
		{  5}\\\hline
	\end{tabular}
	\\
	\begin{tabular}{|*{14}{>{\centering\arraybackslash}m{0.3cm}|}}
		\hline
		{10} &
		{  5} &
		{ 12} &
		{ 15} &
		{  4} &
		{  8} &
		{  1} &
		{\cellcolor{gray!25}3} &
		{  7} &
		{ 12} &
		{ 11} &
		{  4} &
		{  6} &
		{  5}\\\hline
	\end{tabular}
	\\
	\begin{tabular}{|*{14}{>{\centering\arraybackslash}m{0.3cm}|}}
		\hline
		{10} &
		{  5} &
		{ 12} &
		{ 15} &
		{  4} &
		{  8} &
		{\cellcolor{gray!25}1} &
		{  3} &
		{  7} &
		{ 12} &
		{ 11} &
		{  4} &
		{  6} &
		{  5}\\\hline
	\end{tabular}
	\\
	\begin{tabular}{|*{14}{>{\centering\arraybackslash}m{0.3cm}|}}
		\hline
		{10} &
		{  5} &
		{ 12} &
		{ 15} &
		{  4} &
		{\cellcolor{gray!25}1} &
		{  8} &
		{  3} &
		{  7} &
		{ 12} &
		{ 11} &
		{  4} &
		{  6} &
		{  5}\\\hline
	\end{tabular}
	\\
	\begin{tabular}{|*{14}{>{\centering\arraybackslash}m{0.3cm}|}}
		\hline
		{10} &
		{  5} &
		{ 12} &
		{ 15} &
		{\cellcolor{gray!25}1} &
		{  4} &
		{  8} &
		{  3} &
		{  7} &
		{ 12} &
		{ 11} &
		{  4} &
		{  6} &
		{  5}\\\hline
	\end{tabular}
	\\
	\begin{tabular}{|*{14}{>{\centering\arraybackslash}m{0.3cm}|}}
		\hline
		{10} &
		{  5} &
		{ 12} &
		{\cellcolor{gray!25}1} &
		{ 15} &
		{  4} &
		{  8} &
		{  3} &
		{  7} &
		{ 12} &
		{ 11} &
		{  4} &
		{  6} &
		{  5}\\\hline
	\end{tabular}
	\\
	\begin{tabular}{|*{14}{>{\centering\arraybackslash}m{0.3cm}|}}
		\hline
		{10} &
		{  5} &
		{\cellcolor{gray!25}1} &
		{ 12} &
		{ 15} &
		{  4} &
		{  8} &
		{  3} &
		{  7} &
		{ 12} &
		{ 11} &
		{  4} &
		{  6} &
		{  5}\\\hline
	\end{tabular}
	\\
	\begin{tabular}{|*{14}{>{\centering\arraybackslash}m{0.3cm}|}}
		\hline
		{10} &
		{\cellcolor{gray!25}1} &
		{  5} &
		{ 12} &
		{ 15} &
		{  4} &
		{  8} &
		{  3} &
		{  7} &
		{ 12} &
		{ 11} &
		{  4} &
		{  6} &
		{  5}\\\hline
	\end{tabular}
	\\
	\begin{tabular}{|*{14}{>{\centering\arraybackslash}m{0.3cm}|}}
		\hline
		\cellcolor{gray!25}1 & 
		10 & 
		5 & 
		12 & 
		15 &
		4 &
		8 &
		3 &
		7 &
		12 &
		11 &
		4 &
		6 &
		5
		\\\hline
	\end{tabular}

	Ceci achève le premier parcours. On recommence à présent le même
	processus dans le sous-tableau débutant au deuxième élément, ce qui
	donnera le contenu suivant~:

	\begin{tabular}{|*{14}{>{\centering\arraybackslash}m{0.3cm}|}}
		\hline
		\cellcolor{gray!25}1 & \cellcolor{gray!25}3 & 10 & 5 & 12 & 15 & 4 & 8 & 4 & 7 & 12 & 11 & 5 & 6
		\\\hline
	\end{tabular}

	Et ainsi de suite.
	Au total, il y aura $n-1$ étapes.

	\paragraph{Algorithme.}

	Dans cet algorithme, la variable \lda{indiceBulle} est à
	la fois le compteur d’étapes et aussi l’indice de l’élément récepteur
	de la bulle à la fin d’une étape donnée.
	
	\begin{LDA}
	\LComment Trie le tableau reçu en paramètre (via un tri bulle).
		\Algo{triBulle}{\Par{tab\InOut}{\Array{n}{entiers}}}{}
			\Decl{indiceBulle, i}{entiers}
			\For{indiceBulle}{1}{n-1}
				\For[-1]{i}{n – 1}{indiceBulle}
					\If{tab[i] > tab[i + 1]}
						\Stmt swap( tab[i], tab[i + 1] )
						\RComment voir algorithme précédent
					\EndIf
				\EndFor
			\EndFor
		\EndAlgo
	\end{LDA}

%=========================
\section{Cas particuliers}
%==========================

	{\sffamily\bfseries\upshape
	Tri partiel}

		Parfois, on n’est pas intéressé par un tri complet de l’ensemble mais on
		veut uniquement les «~$k$~» plus petites (ou plus grandes) valeurs. Le
		tri par recherche des minima et le tri bulle sont particulièrement
		adaptés à cette situation ; il suffit de les arrêter plus tôt. Par
		contre, le tri par insertion est inefficace car il ne permet pas un
		arrêt anticipé.

	{\sffamily\bfseries\upshape
	Sélection des meilleurs}

		Une autre situation courante est la suivante~: un ensemble de valeurs
		sont traitées (lues, calculées\dots) et il ne faut garder que les
		$k$ plus petites (ou plus grandes).
		L'algorithme est plus simple à écrire si on utilise
		une position supplémentaire en fin de tableau. Notons aussi qu’il faut
		tenir compte du cas où il y a moins de valeurs que le nombre de valeurs
		voulues ; c’est pourquoi on ajoute une variable indiquant le nombre
		exact de valeurs dans le tableau.

		Exemple~:
		on lit un ensemble de valeurs strictement positives (un 0 indique la fin
		de la lecture) et on ne garde que les $k$ plus petites valeurs.

		\begin{LDA}
			\Algo{meilleurs}{\Par{plusPetits\InOut}{\Array{k+1}{entiers}}, \Par{nbValeurs\Out}{entier}}{}
				\Decl{val}{entier}
				\Let nbValeurs \Gets 0
				\Read val
				\While{val ${\geq}$ 0}
					\Stmt insérer( val, plusPetits, nbValeurs )
					\Read val
				\EndWhile
			\EndAlgo
		\end{LDA}
		
		\begin{LDA}			
			\Algo{insérer}{\Par{val\In}{entier}, \Par{plusPetits\InOut}{\Array{k+1}{entiers}}, \Par{nbValeurs\Out}{entier}}{}
				\Decl{i}{entier}
				\Let i \Gets nbValeurs - 1
				\While{i ${\geq}$ 0 ET val < plusPetits[i]}
					\Let plusPetits[i + 1] \Gets plusPetits[i]
					\Let i \Gets i - 1
				\EndWhile
				\Let plusPetits[i + 1] \Gets val
				\If{nbValeurs < k}
					\Let nbValeurs \Gets nbValeurs + 1
				\EndIf
			\EndAlgo
		\end{LDA}

%====================
\section{Références}
%====================

	\begin{itemize}
		\item {
			\url{http://interstices.info/jcms/c_6973/les-algorithmes-de-tri}

			\textit{Ce site permet de visualiser les méthodes de tris en
			fonctionnement. Il présente tous les tris vus en première plus quelques
			autres comme le «~tri rapide~» («~quicksort~») que vous verrez en
			deuxième année. }}
	\end{itemize}
