\chapter{Le LDA}

	Dans cette annexe nous définissons le LDA
	(\emph{le Langage de Description d'Algorithmes})
	que nous allons utiliser. 
	Nous ne nous attarderons pas sur les concepts
	ni sur certaines bonnes pratiques;
	tout cela est vu dans les chapitres associés.
	
	Nous utilisons un pseudo-code pour nous libérer 
	des contraintes des langages de programmation.
	\begin{itemize}
	\item
		Un programme est une suite de lignes
		ne permettant pas d'utiliser pleinement les
		deux dimensions de la page 
		(pensons à la mise en page des formules).
	\item
		Certaines constructions et règles
		n'existent que pour simplifier le travail du compilateur
		et/ou accélérer le code.
		C'est le cas, par exemple, de la syntaxe du \emph{switch}.
	\end{itemize}
	
	Dans vos réflexions, brouillons, premiers jets,
	nous vous encourageons à utiliser des notations
	qui vous sont propres et qui vous permettent
	de poser votre réflexion sur un papier
	et d'avancer vers une solution.
	
	La version finale, toutefois, 
	doit être lue par d'autres personnes.
	Il est \textbf{essentiel} qu'il n'y ait aucune ambigüité
	sur le sens de votre écrit.
	C'est pourquoi, nous devons définir une notation
	à la fois souple et précise.
	
	Cette notation doit aussi être adaptée à des étudiants de première année.
	Ce qui nous amène à ne pas introduire des nuances qui leur échappent encore
	et, parfois, à imposer des contraintes qui seront relâchées plus tard
	mais qui permettent de cadrer l'apprentissage d'un débutant.
	
	\textbf{Remarque} : Ce guide n'est pas universel.
	En dehors de l'école, d'autres notations sont utilisées,
	parfois proches, parfois plus lointaines.
	Votre professeur pourra également introduire quelques notations
	qui ne sont pas reprises ici 
	ou relâcher quelques contraintes définies ici. 
	Lorsque vous changerez de professeur,
	soyez conscient que ces ajouts ne seront peut-être plus valables.
	
	L'important est que le groupe qui doit communiquer
	au moyen d'algorithmes se soit préalablement mis d'accord 
	sur des notations.

\subsection*{Un algorithme}
%---------------------------------

	\begin{minipage}{6cm}
		\begin{LDA}
			\Algo{nom}{paramètres}{Type}
				\Stmt Instructions
				\Return expression
			\EndAlgo
		\end{LDA}
	\end{minipage}
	\qquad
	\begin{minipage}{6cm}
		\begin{LDA}
			\Algo{nom}{paramètres}{}
				\Stmt Instructions
			\EndAlgo
		\end{LDA}
	\end{minipage}

	On permet l'utilisation du raccourci \lda{\K{algo}}.
	
\subsection*{Types, variables et constantes}
%-----------------------------------

	Les types permis sont :
	\lda{entier}, \lda{réel},
	\lda{booléen} et \lda{chaine}.

	\begin{LDA}
		\Const{nom}{valeur}
		\Decl{var1, \dots}{Type} 
	\end{LDA}

\subsection*{Les instructions de base}
%---------------------------------

	\begin{LDA}
		\Let var \Gets expression
		\Read var1, var2\dots
		\Write expression1, expression2\dots
		\Error "raison" \Comment Provoque l'arrêt de l'algorithme.
	\end{LDA}
	
\subsection*{Les instructions de choix}
%---------------------------------

	\begin{minipage}[t]{4.5cm}
	\begin{LDA}
	\If{condition}
		\Stmt Instructions
	\EndIf
	\Empty
	\Empty
	\Empty
	\Empty
	\Empty
	\Empty
	\end{LDA}
	\end{minipage}
	\ 
	\begin{minipage}[t]{4.5cm}
	\begin{LDA}
	\If{condition}
		\Stmt Instructions
	\Else
		\Stmt Instructions
	\EndIf
	\Empty
	\Empty
	\Empty
	\Empty
	\end{LDA}
	\end{minipage}
	\ 
	\begin{minipage}[t]{4.5cm}
	\begin{LDA}
	\If{condition}
		\Stmt Instructions
	\ElsIf{condition}
		\Stmt Instructions
	\ElsIf{condition}
		\Stmt \dots
	\Else
		\Stmt Instructions
	\EndIf
	\end{LDA}
	\end{minipage}
	
	\begin{LDA}
		\Switch{expression \K{vaut}}
			\Case{liste${}_1$ de valeurs séparées par des virgules }
				\Stmt Instructions
			\Case{liste${}_2$ de valeurs séparées par des virgules }
				\Stmt Instructions
			\Empty \dots
			\Case{liste${}_k$ de valeurs séparées par des virgules }
				\Stmt Instructions
			\Case{\K{autres }}
				\Stmt Instructions
		\EndSwitch
	\end{LDA}
	
	où l'expression peut être de type \lda{entier} ou \lda{chaine}
	(pas de \lda{réel}) et les valeurs sont des constantes.

\subsection*{Les instructions de répétition}
%---------------------------------

	\begin{minipage}[t]{4cm}	
	\begin{LDA}
		\While{condition}
			\Stmt Instructions
		\EndWhile
	\end{LDA}
	\end{minipage}
	\
	\begin{minipage}[t]{6.5cm}	
	\begin{LDA}
		\For{indice \K{de} début \K{à} fin [\K{par} pas]}
			\Stmt Instructions
		\EndFor
	\end{LDA}
	\end{minipage}
	\
	\begin{minipage}[t]{4cm}	
	\begin{LDA}
		\Repeat
			\Stmt Instructions
		\EndRepeat{condition}
	\end{LDA}
	\end{minipage}
		
	La boucle \lda{pour} ne peut être utilisée que pour des entiers.

	Il n'est \textbf{pas nécessaire} de déclarer l'indice.
	Il ne peut être utilisé en dehors de la boucle et ne peut pas
	être modifié à l'intérieur de la boucle.
	De même, le \lda{début}, la \lda{fin} et le \lda{pas} 
	ne peuvent pas être modifiés dans la boucle.
