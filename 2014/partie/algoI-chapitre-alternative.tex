\chapter{Les alternatives}
% ========================

\marginicon{objectif}
Dans ce chapitre, nous abordons les structures alternatives qui
permettent de conditionner des parties d’algorithmes.
Elles ne seront exécutées que si une condition est satisfaite.

\section{«~si~–~alors~–~sinon~»}
% ==============================

	Cette structure permet d’exécuter une partie de code ou
	une autre en fonction de la valeur de vérité d’une condition.
	
	\begin{Emphase}[definition]{si – alors}
	\begin{Pseudocode}
	\If{condition}
		\LComment instructions à réaliser si la condition est VRAIE
	\EndIf
	\end{Pseudocode}
	\end{Emphase}
	
	La \textbf{condition} est une expression délivrant un résultat booléen
	(\textbf{vrai} ou \textbf{faux}) ; 
	elle associe des variables, constantes, expressions arithmétiques, 
	au moyen des opérateurs logiques ou de comparaison. 
	En particulier, cette condition peut être réduite à
	une seule variable booléenne.
	
	Dans cette structure, lorsque la condition est vraie,
	il y a exécution de la séquence d’instructions contenue 
	entre les mots \pseudocode{\K{alors}} et \pseudocode{\K{fin si}} ; 
	ensuite, l’algorithme continue de façon séquentielle.
	
	Lorsque la condition est fausse, les instructions se trouvant entre 
	\pseudocode{\K{alors}} et \pseudocode{\K{fin si}} 
	sont tout simplement ignorées.
	
	\begin{Emphase}[definition]{si – alors – sinon}
		\begin{Pseudocode}
		\If{condition}
			\LComment instructions à réaliser si la condition est VRAIE
		\Else
			\LComment instructions à réaliser si la condition est FAUSSE
		\EndIf
		\end{Pseudocode}
	\end{Emphase}
	
	Dans cette structure, une et une seule des deux séquences est exécutée.
	
	\begin{Emphase}[exercice]{Exemple~:~Signe d’un nombre}
		Écrire un algorithme qui affiche si un nombre lu est positif 
		(zéro inclus) ou strictement négatif.
		
		{\bfseries Solution}
		
		\begin{Pseudocode}
		\LComment Lit un nombre et affiche si ce nombre est positif (zéro inclus)
		ou strictement négatif
		\Module{signeNombre}{}{}
			\Decl nb : entier
			\Read nb
			\If{nb < 0}
				\Write "le nombre", nb, " est négatif"
			\Else
				\Write "le nombre", nb, " est positif ou nul"
			\EndIf
		\EndModule
		\end{Pseudocode}
	\end{Emphase}
	
	
	\begin{Emphase}[exercice]{Exercice~:~Signe d’un nombre (amélioré)}
		Écrire un algorithme qui dit si un nombre lu est positif, 
		négatif ou nul.
	\end{Emphase}

\section{Les conditions}
% ==============================

	Quelles sont les possibilités pour écrire la condition
	d'une alternative ?
	On peut y mettre toute expression à valeur booléenne,
	c'est-à-dire qui donnera \pseudocode{vrai} ou \pseudocode{vrai}
	comme résultat.
	
	Pour cela, on peut utiliser des opérateurs de comparaison
	et des opérateurs logiques.
	
	\subsection{Opérateurs de comparaison}

		Ces opérateurs agissent généralement sur des variables numériques ou des
		chaines et donnent un résultat booléen.

		\begin{center}
		\begin{tabular}{m{2cm}|m{11cm}}
		\raggedleft  \pseudocode{=} & égal\\
		\raggedleft  \pseudocode{{\textless}{\textgreater}}
			ou \pseudocode{${\neq}$} &  différent de\\
		\raggedleft  \pseudocode{\textless} & (strictement) plus petit que\\
		\raggedleft  \pseudocode{\textgreater} & (strictement) plus grand que\\
		\raggedleft  \pseudocode{${\leq}$} & plus petit ou égal\\
		\raggedleft  \pseudocode{${\geq}$} & plus grand ou égal\\
		\end{tabular}
		\end{center}

		Pour les chaines, 
		c’est l’ordre alphabétique qui détermine le résultat 
		(par exemple
		\pseudocode{"milou" < "tintin"} est \textbf{vrai} 
		de même que
		\pseudocode{"assembleur" ${\leq}$ "java"})

	\subsection{Opérateurs logiques}

		Ils agissent sur des expressions booléennes (variables ou expressions à
		valeurs booléennes) pour donner un résultat du même type.

		\begin{center}
		\begin{tabular}{m{1cm}|m{12cm}}
		\raggedleft \pseudocode{NON} & négation\\
		\raggedleft \pseudocode{ET} & conjonction logique\\
		\raggedleft \pseudocode{OU} & disjonction logique\\
		\end{tabular}
		\end{center}

		Pour rappel, \pseudocode{cond1 ET cond2} n’est vrai que lorsque
		les deux conditions sont vraies. \pseudocode{cond1 OU cond2} est
		toujours vrai, sauf quand les deux conditions sont fausses.

		Veillez à mettre des parenthèses dans le cas de combinaisons de ET et de
		OU~: \pseudocode{(cond1 ET cond2) OU cond3} étant différent de
		\pseudocode{cond1 ET (cond2 OU cond3).} En cas
		d’oubli de parenthèses, il faudra se rappeler que
		\pseudocode{ET} est prioritaire sur le \pseudocode{OU}.
		
		Pour rappel aussi, pour un booléen \pseudocode{ok}~: 
		\pseudocode{ok = faux} est équivalent à \pseudocode{NON ok},
		\pseudocode{ok = vrai} est équivalent à \pseudocode{ok} et 
		\pseudocode{NON NON ok} est équivalent à \pseudocode{ok}.
		Dans les trois cas, nous préconiserons la seconde écriture.

	\subsection{Évaluation complète et court-circuitée}

		On définit deux modes d’évaluation des opérateurs \pseudocode{ET}
		et \pseudocode{OU}~:

		\begin{description}
		\item[L’évaluation \textit{complète}]
			pour connaitre la valeur de
			\pseudocode{cond1 ET cond2} (respectivement
			\pseudocode{cond1 OU cond2}), les deux conditions sont chacune
			évaluées, après quoi on évalue la valeur de vérité de l’ensemble de
			l’expression.
		\item[L’évaluation \textit{court-circuitée}]
			dans un premier temps, seule la
			première condition est testée. Dans le cas du \pseudocode{ET},
			si \pseudocode{cond1} s’avère faux, il est inutile d’évaluer
			\pseudocode{cond2} puisque le résultat sera faux de toute façon
			; l’évaluation de \pseudocode{cond2} et de l’ensemble de la
			conjonction ne se fera que si \pseudocode{cond1} est vrai. De
			même, dans le cas du \pseudocode{OU}, si
			\pseudocode{cond1} s’avère vrai, il est inutile d’évaluer
			\pseudocode{cond2} puisque le résultat sera vrai de toute façon
			; l’évaluation de \pseudocode{cond2} et de l’ensemble de la
			disjonction ne se fera que si \pseudocode{cond1} est faux.
		\end{description}

		Dans le cadre de ce cours, nous opterons pour la deuxième 
		interprétation.
		Montrons son avantage sur un exemple.
		Considérons	l’expression 
		\pseudocode{n }\pseudocode{${\neq}$}\pseudocode{ 0
		ET m/n {\textgreater} 10}. Si on teste sa valeur de vérité avec une
		valeur de \pseudocode{n} non nulle, la première condition est
		vraie et le résultat de la conjonction dépendra de la valeur de la
		deuxième condition. Supposons à présent que \pseudocode{n} soit
		nul. L’évaluation court-circuitée donne le résultat \textbf{faux}
		immédiatement après test de la première condition sans évaluer la
		seconde, tandis que l’évaluation complète entrainerait un arrêt de
		l’algorithme pour cause de division par 0~!

		Notez que l’évaluation court-circuitée a pour conséquence la
		non-commutativité du \pseudocode{ET} et du
		\pseudocode{OU~}: \pseudocode{cond1 ET cond2} n’est donc
		pas équivalent à \pseudocode{cond2 ET cond1}, puisque l’ordre
		des évaluations des deux conditions entre en jeu.~Nous conseillons
		cependant de limiter les cas d’utilisation de l’évaluation
		court-circuitée et d’opter pour des expressions dont
		l’évaluation serait similaire dans les deux cas. La justification
		d’utiliser l’évaluation court-circuitée apparaitra dans plusieurs
		exemples tout au long du cours.
	
\section{Indentation}
% ==============================
	
	Dans l’écriture de tout algorithme, on veillera à \textbf{indenter}
	correctement les lignes de codes afin de faciliter sa lecture ; cela
	veut dire que~:
	
	\begin{liste}
	\item
		Les \textbf{balises} encadrant toute structure de contrôle 
		devront être parfaitement à la verticale l’une de l’autre~:~
		\pseudocode{\K{module}} et \pseudocode{\K{fin module}} ; 
		\pseudocode{\K{si}} [, \pseudocode{\K{sinon}}] 
		et \pseudocode{\K{fin si}}.
		Ce sera aussi vrai pour celles que nous allons voir plus tard~: 
		\emph{selon que}, \emph{tant que}, \emph{faire jusqu’à ce},
		\emph{pour}\dots	
	\item
		Les lignes situées entre toute paire de balises 
		devront être décalées d’une tabulation vers la droite.
	\item
		On pensera aussi à tracer une \textbf{ligne verticale} 
		entre le début et la fin d’une structure de contrôle 
		afin de mieux la délimiter encore 
		(surtout lorsqu’on travaille sur papier).
	\end{liste}
	
	\section{«~selon que~»}
	
	Avec ces structures, plusieurs branches d’exécution
	sont disponibles. L’ordinateur choisit la branche à
	exécuter en fonction de la valeur d’une variable 
	(ou parfois d’une expression) ou de
	la condition qui est vraie.
	
	\begin{Emphase}[definition]{selon que (version avec listes de valeurs)}

	\begin{Pseudocode}
		\Switch{variable \K{vaut}}
			\Case{liste\_1 de valeurs séparées par des virgules }
				\LComment instructions lorsque la valeur de la variable est dans liste\_1
			\Case{liste\_2 de valeurs séparées par des virgules }
				\LComment instructions lorsque la valeur de la variable est dans liste\_2		
			\Empty \dots
			\Case{liste\_n de valeurs séparées par des virgules }
				\LComment instructions lorsque la valeur de la variable est dans liste\_n
			\Case{\K{autres }}
				\LComment instructions lorsque la valeur de la variable 
				\LComment ne se trouve dans aucune des listes précédentes
		\EndSwitch
	\end{Pseudocode}

	\end{Emphase}
	
	Dans ce type de structure, comme pour la structure
	\textbf{si-alors-sinon}, une seule des séquences d’instructions
	sera exécutée. On veillera à ne pas faire apparaitre une même valeur
	dans plusieurs listes. Cette structure est une simplification
	d’écriture de plusieurs alternatives imbriquées. Elle est équivalente à~:
	
	\begin{Pseudocode}
	\If{variable = une des valeurs de la liste\_1}
		\LComment instructions lorsque la valeur est dans liste\_1
	\Else
		\If{variable = une des valeurs de la liste\_2}
			\LComment instructions lorsque la valeur est dans liste\_2
		\Else
			\Empty\dots
			\If{variable = une des valeurs de la liste\_n}
				\LComment instructions lorsque la valeur est dans liste\_n
			\Else
				\LComment instructions lorsque la valeur de la variable
				\LComment ne se trouve dans aucune des listes précédentes
			\EndIf
		\EndIf
	\EndIf
	\end{Pseudocode}
	
	Notez que le cas \pseudocode{\K{autres}} est facultatif.
	
	\begin{Emphase}[definition]{selon que (version avec conditions)}
	\begin{Pseudocode}
	\Switch{}
		\Case{condition\_1 }
			\LComment instructions lorsque la condition\_1 est vraie
		\Case{condition\_2 }
			\LComment instructions lorsque la condition\_2 est vraie
		\Empty \dots
		\Case{condition\_n }
			\LComment instructions lorsque la condition\_n est vraie
		\Case{\K{autres }}
			\LComment instructions à exécuter quand aucune
			\LComment des conditions précédentes n’est vérifiée
	\EndSwitch
	\end{Pseudocode}
	\end{Emphase}
	
	Comme précédemment, une et une seule des séquences d’instructions est
	exécutée. On veillera à ce que les conditions ne se «~recouvrent~» pas,
	c’est-à-dire que deux d’entre elles ne soient jamais vraies
	simultanément. C’est équivalent à~:
	
	\begin{Pseudocode}
	\If{condition\_1}
		\LComment instructions lorsque la condition\_1 est vraie
	\Else
		\If{condition\_2}
			\LComment instructions lorsque la condition\_2 est vraie
		\Else
			\Empty \dots
			\If{condition\_n}
				\LComment instructions lorsque la condition\_n est vraie
			\Else
				\LComment instructions à exécuter quand aucune
				\LComment des conditions précédentes n’est vérifiée
			\EndIf
		\EndIf
	\EndIf
	\end{Pseudocode}
	
	\begin{Emphase}[exercice]{Exemple~:~Jour de la semaine en clair}
	Écrire un algorithme qui lit un jour de la semaine sous forme
	d’un nombre entier 
	(1 pour lundi, \dots, 7 pour dimanche) 
	et qui affiche en clair ce jour de la semaine.
	
	{\bfseries Solution}
	
	\begin{Pseudocode}
	\LComment Lit un nombre entre 1 et 7 et affiche en clair le jour de la semaine correspondant.
	\Module{jourSemaine}{}{}
	\Decl jour : entier
	\Read jour
		\Switch{jour \K{vaut}}
			\Stmt 1 : \K{afficher} "lundi"
			\Stmt 2 : \K{afficher} "mardi"
			\Stmt 3 : \K{afficher} "mercredi"
			\Stmt 4 : \K{afficher} "jeudi"
			\Stmt 5 : \K{afficher} "vendredi"
			\Stmt 6 : \K{afficher} "samedi"
			\Stmt 7 : \K{afficher} "dimanche"
		\EndSwitch
	\EndModule
	\end{Pseudocode}
	\end{Emphase}
	
	
	\begin{Emphase}[exercice]{Exemple~:~Nombre de jours (avec énumération)}
	
	Reprendre l’algorithme qui affiche le nombre de jours
	dans un mois en utilisant une énumération.
	
	{\bfseries Solution}
	
	\begin{Pseudocode}
	\footnotesize
	\Stmt \K{énumération} Mois \{JANVIER, FÉVRIER, MARS, AVRIL, MAI, JUIN, JUILLET, AOÛT, SEPTEMBRE, OCTOBRE, NOVEMBRE, DÉCEMBRE\}
	\end{Pseudocode}
	
	\begin{Pseudocode}
	\LComment Lit un Mois et affiche le nombre de jours correspondant
	\LComment (en ne tenant pas compte des années bissextiles).
	\Module{nbJours}{}{}
		\Decl unMois : Mois
		\Read unMois
		\RComment on lira la valeur JANVIER ou FÉVRIER ou \dots{} ou DÉCEMBRE
		\Switch{unMois \K{vaut}}
			\Case{JANVIER, MARS, MAI, JUILLET, AOÛT, OCTOBRE, DÉCEMBRE}
				\Write 31
			\Case{AVRIL, JUIN, SEPTEMBRE, NOVEMBRE}
				\Write 30
			\Case{FÉVRIER} \RComment{on ne tient pas compte ici des années bissextiles}
				\Write 28
		\EndSwitch
		\LComment{Pas de clause \og{}autre\fg{} car toutes les valeurs possibles ont été envisagées}
	\EndModule
	\end{Pseudocode}
	\end{Emphase}

\section{Exercices}
% ================================

\begin{Exercice}{Compréhension}

	Qu’affichent les algorithmes suivants si à chaque fois les deux nombres
	lus au départ sont, dans l’ordre, 2 et 3~? Même question avec 4 et 1.
	
	\begin{Pseudocode}
	\Module{exerciceA}{}{}
		\Decl a,b : entier
		\Read a,b
		\If{a > b}
			\Let a \Gets a + 2 * b
		\EndIf
		\Write a
	\EndModule
	\end{Pseudocode}

	\begin{Pseudocode}
	\Module{exerciceB}{}{}
		\Decl a,b,c : entier
		\Read b,a
		\If{a > b}
			\Let c \Gets a DIV b
		\Else
			\Let c \Gets b MOD a	
		\EndIf
		\Write c
	\EndModule
	\end{Pseudocode}

	\begin{Pseudocode}
	\Module{exerciceC}{}{}
		\Decl x1,x2 : entier
		\Decl ok : booléen
		\Read x1,x2
		\Let ok \Gets x1 > x2
		\If{ok}
			\Let ok \Gets ok ET x1 = 4
		\Else
			\Let ok \Gets ok OU x2 = 3
		\EndIf
		\If{ok}
			\Let x1 \Gets x1 * 1000
		\EndIf
		\Write x1 + x2
	\EndModule
	\end{Pseudocode}
\end{Exercice}

\begin{Exercice}{Simplification d’algorithmes}
	Voici quelques extraits d’algorithmes corrects du point de vue de la
	syntaxe mais contenant des lignes inutiles ou des lourdeurs d’écriture.
	Remplacer chacune de ces portions d’algorithme par un minimum
	d’instructions qui auront un effet équivalent.

	\begin{Pseudocode}
		\Let ok2 \Gets NON (ok1 = vrai) ET NON (ok1 = faux)
	\end{Pseudocode}
	
	\begin{Pseudocode}
	\If{ok = vrai}
		\Write nombre
	\EndIf
	\end{Pseudocode}

	\begin{Pseudocode}
	\If{ok = faux}
		\Write nombre
	\EndIf
	\end{Pseudocode}

	\begin{Pseudocode}
	\If{condition}
		\Let ok \Gets vrai
	\Else
		\Let ok \Gets faux
	\EndIf
	\end{Pseudocode}

	\begin{Pseudocode}
	\If{a $>$ b}
		\Let ok \Gets faux
	\Else
		\If{a $\leq$ b}
			\Let ok \Gets vrai
		\EndIf
	\EndIf
	\end{Pseudocode}

	\begin{Pseudocode}
	\If{ok1 = vrai ET ok2 = vrai}
		\Write x
	\EndIf
	\end{Pseudocode}
	
\end{Exercice}

\begin{Exercice}{Maximum de 2 nombres}
	\marginicon{java}
	Écrire un algorithme qui, étant donné deux nombres quelconques,
	recherche et affiche le plus grand des deux. Attention~! On ne veut
	pas savoir si c’est le premier ou le deuxième qui est
	le plus grand mais bien quelle est cette plus grande valeur. Le
	problème est donc bien défini même si les deux nombres sont
	identiques.
\end{Exercice}

\begin{Exercice}{Maximum de 3 nombres}
	\marginicon{java}
	Écrire un algorithme qui, étant donné trois nombres quelconques,
	recherche et affiche le plus grand des trois.
\end{Exercice}

\begin{Exercice}{Le signe}
	\marginicon{java}
	Écrire un algorithme qui affiche un message indiquant
	si un entier est strictement négatif, nul ou strictement
	positif.
\end{Exercice}

\begin{Exercice}{La fourchette}
	Écrire un algorithme qui, étant donné trois nombres, 
	recherche et affiche si le premier des trois 
	appartient à l’intervalle donné par le plus petit et le plus grand 
	des deux autres (bornes exclues). 
	Qu’est-ce qui change si on inclut les bornes~?
\end{Exercice}

\begin{Exercice}{Équation du deuxième degré}
	Écrire un algorithme qui, 
	étant donné une équation du deuxième degré, 
	déterminée par le coefficient de \textit{x}\up{2},
	le coefficient de \textit{x} et le terme indépendant, 
	recherche et affiche la (ou les) racine(s) de l’équation 
	(ou un message adéquat s’il n’existe pas de racine réelle).
\end{Exercice}

\begin{Exercice}{Une petite minute}
	Écrire un algorithme qui, à partir d’un moment exprimé par 2 entiers, 
	heure et minute, affiche le moment qu’il sera une minute plus tard.
\end{Exercice}

\begin{Exercice}{Calcul de salaire}
	Dans une entreprise, 
	une retenue spéciale de 15\% est pratiquée 
	sur la partie du salaire mensuel qui dépasse 1200~\euro. 
	Écrire un algorithme qui calcule le salaire net à partir du salaire brut. 
	En quoi l’utilisation de constantes convient-elle pour améliorer cet algorithme~?
\end{Exercice}

\begin{Exercice}{Nombre de jours dans un mois}
	Écrire un algorithme qui donne le nombre de jours dans un mois. 
	Le mois est lu sous forme d’un entier (1 pour janvier\dots).
	On considère dans cet exercice que le mois de février
	comprend toujours 28 jours.
\end{Exercice}

\begin{Exercice}{Année bissextile}
	\marginicon{java}
	Écrire un algorithme qui vérifie si une année est bissextile. 
	Pour rappel, les années bissextiles sont les années multiples de 4. 
	Font exception, les multiples de 100 
	(sauf les multiples de 400 qui sont bien bissextiles). 
	Ainsi $2012$ et $2400$ sont bissextile mais pas $2010$ ni $2100$.
\end{Exercice}

\begin{Exercice}{Valider une date}
	Écrire un algorithme qui valide une date donnée par trois entiers~:~
	l’année, le mois et le jour.
\end{Exercice}

\begin{Exercice}{Le jour de la semaine}
	\marginicon{java}
	Écrire un algorithme qui lit un jour du mois de novembre de cette année
	(sous forme d’un entier entre 1 et 30) 
	et qui affiche le nom du jour («~lundi~», «~mardi~»\dots)
\end{Exercice}

\begin{Exercice}{Quel jour serons-nous~?}
	Écrire un algorithme qui indique le nom du jour 
	qui sera \textit{n} jours plus tard qu’un jour donné. 
	Par exemple si le jour donné est «~mercredi~» et \textit{n} = 10, 
	l’algorithme indiquera «~samedi~».
\end{Exercice}

\begin{Exercice}{Un peu de trigono}
	Écrire un algorithme qui pour un entier \textit{n} donné, 
	affiche la valeur de $cos(n * \pi/2)$.
\end{Exercice}

\begin{Exercice}{Le stationnement alternatif}
	Dans une rue où se pratique le stationnement alternatif, 
	du 1 au 15 du mois, on se gare du côté des maisons ayant un numéro impair, 
	et le reste du mois, on se gare de l’autre côté. 
	Écrire un algorithme qui, sur base de la date du jour et du numéro de maison
	devant laquelle vous vous êtes arrêté, 
	indique si vous êtes bien stationné ou non.
\end{Exercice}
