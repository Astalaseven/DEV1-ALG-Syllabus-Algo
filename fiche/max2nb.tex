%======================================
\begin{Fiche}{Maximum de deux nombres}
%======================================
\label{fiche:max2nb}

	Quel est le maximum de deux nombres ?

\Section{Analyse}

	Voilà un classique de l'algorithmique.
	Attention ! On ne veut pas savoir \emph{lequel}
	est le plus grand mais juste la valeur.
	Il n'y a donc pas d'ambigüité si les deux nombres sont égaux.

	\textbf{Données} : deux nombres réels.
		
	\textbf{Résultat} : un réel contenant la plus grande des deux valeurs données.

	\begin{center}	
		\flowalgodd{nb1 (réel)}{nb2 (réel)}{max2}{réel}
	\end{center}

\Section{Exemples}

	\vspace*{-3mm}
	\begin{multicols}{3}
		\begin{itemize}
		\item \lda{max2(-3, 4)} donne $4$
		\item \lda{max2(7, 4)} donne $7$
		\item \lda{max2(4, 4)} donne $4$
		\end{itemize}
	\end{multicols}
	\vspace*{-6mm}
	
\Section{Solution}

	\begin{LDA}
	\Algo{max2}{\Par{nb1}{réel}, \Par{nb2}{réel}}{réel}
		\Decl{max}{réel}
		\If{nb1 > nb2}
			\Let max \Gets nb1
		\Else
			\Let max \Gets nb2
		\EndIf
		\Return max
	\EndAlgo
	\end{LDA}

\Section{Alternatives}

	\begin{minipage}{7.0cm}
		\begin{LDA}
		\Algo{max2}{\Par{nb1}{réel}, \Par{nb2}{réel}}{réel}
			\If{nb1 > nb2}
				\Return nb1
			\Else
				\Return nb2
			\EndIf
		\EndAlgo
		\end{LDA}
	\end{minipage}
	\quad
	\begin{minipage}{6.5cm}
		En algorithmique, comme ailleurs, il existe des modes.
		Certaines personnes insistent pour qu'il n'y ait qu'un seul
		retour en fin d'algorithme ; 
		d'autres admettent un retour à la fin de chaque
		branche de l'alternative.
		En début d'apprentissage,
		on vous demande de n'utiliser qu'un seul retour
		pour éviter tout abus.
	\end{minipage}

	\hfil\rule{0.5\textwidth}{.4pt}\hfil

	\begin{minipage}{7.0cm}
		\begin{LDA}
		\Algo{max2}{\Par{nb1}{réel}, \Par{nb2}{réel}}{réel}
			\Decl{max}{réel}
			\Let max \Gets nb1
			\If{nb2 > nb1}
				\Let max \Gets nb2
			\EndIf
			\Return max
		\EndAlgo
		\end{LDA}
	\end{minipage}
	\quad
	\begin{minipage}{6.5cm}
		Certains écrivent parfois une solution de ce genre
		mais ne la défendent pas avec les bons arguments.
		Le fait de ne pas avoir de "sinon" n'est absolument pas pertinent.
		Son avantage est qu'elle se généralise plus facilement
		au cas où il y a plusieurs nombres dont on veut le maximum.
		Pour deux nombres, on lui préférera la solution proposée plus haut.
	\end{minipage}
	
\Section{Quand l'utiliser ?}

	Cet algorithme peut bien sûr être facilement adapté
	à la recherche du minimum.
		
\end{Fiche}
