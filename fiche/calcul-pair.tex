\begin{Fiche}{Un nombre (im)pair}
\label{fiche:calcul-pair}

\Section{Le problème}
	Un nombre reçu en paramètre est-il pair~?

\Section{Analyse}

	Un nombre est pair si il est multiple de 2. 
	C'est-à-dire si le reste de sa division par 2 vaut 0.
	\begin{LDA}
		\Stmt estPair est vrai si nombre MOD 2 = 0
	\end{LDA}

	\paragraph{Données}
	\begin{itemize}
	\item le nombre entier dont on veut savoir si il est pair.
	\end{itemize}

	\paragraph{Résultat.}
	Un booléen à \textit{vrai} si le \textit{nombre} est pair et \textit{faux} sinon.

	\bigskip
	\begin{center}	
	\flowalgod{nombre (entier)}{isPair}{booléen}
	\end{center}

\Section{Exemples}

	\begin{itemize}
	\item \lda{isPair(2016)} donne $vrai$
	\item \lda{isPair(2015)} donne $faux$
	\end{itemize}
	
\Section{Solution}

	\begin{LDA}
%	\LComment Retourne vrai si le nombre reçu en paramètre est pair et faux sinon.
%	\LComment Données : le nombre dont on veut savoir si il est pair
%	\LComment Résultat : vrai si le nombre est pair et faux sinon.
	\Algo{isPair}{\Par{nombre}{entier}}{booléen}
		\Return nombre MOD 2 = 0
	\EndAlgo
	\end{LDA}

\Section{Alternatives}

	\begin{minipage}{7.5cm}
		\begin{LDA}
		\Algo{isPair}{\Par{nombre}{entier}}{booléen}
			\If{nombre MOD 2 = 0}
				\Return vrai
			\Else
				\Return faux
			\EndIf
		\EndAlgo
		\end{LDA}
	\end{minipage}
	\quad
	\begin{minipage}{6cm}
		Certains étudiants se sentent plus à l'aise avec
		la solution ci-contre en début d'année.			
		C'est probablement parce qu'elle colle plus à la façon de l'exprimer
		en français.
		On les encourage toutefois à rapidement 
		passer à la version plus compacte
		et, une fois habitué, plus lisible.
	\end{minipage}

	\hfil\rule{0.5\textwidth}{.4pt}\hfil

	\begin{minipage}{7.5cm}
		\begin{LDA}
		\Algo{isPair}{\Par{nombre}{entier}}{booléen}
			\If{nombre MOD 2 = 0}
				\Return vrai
			\EndIf
			\Return faux
		\EndAlgo
		\end{LDA}
	\end{minipage}
	\quad
	\begin{minipage}{6cm}
		On rencontre également ce genre de solution
		qui, pour certains, 
		parait mieux que la précédente 
		parce qu'elle ne contient pas de "sinon"
		et est donc plus courte.
		Il n'en n'est rien.
		Rappelons que la longueur de l'algorithme
		n'est pas, en soi, un critère de qualité. 
	\end{minipage}
	
\Section{Quand l'utiliser ?}

	À chaque fois qu'un résultat booléen dépend d'un calcul simple.
	Si le calcul est plus compliqué, on peut le décomposer comme
	indiqué dans la fiche \vref{fiche:calcul-complexe}.
	
	On peut également s'inspirer de cette solution
	quand il faut donner sa valeur à une variable booléenne.
		
\end{Fiche}
